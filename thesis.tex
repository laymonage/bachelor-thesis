%
% Template Laporan Skripsi/Thesis
%
% @author  Andreas Febrian, Lia Sadita
% @version 1.03
%
% Dokumen ini dibuat berdasarkan standar IEEE dalam membuat class untuk
% LaTeX dan konfigurasi LaTeX yang digunakan Fahrurrozi Rahman ketika
% membuat laporan skripsi. Konfigurasi yang lama telah disesuaikan dengan
% aturan penulisan thesis yang dikeluarkan UI pada tahun 2008.
%

%
% Tipe dokumen adalah report dengan satu kolom.
%
\documentclass[12pt, a4paper, onecolumn, twoside, final]{report}
\raggedbottom

% Load konfigurasi LaTeX untuk tipe laporan thesis
\usepackage{uithesis}


% Load konfigurasi khusus untuk laporan yang sedang dibuat
%-----------------------------------------------------------------------------%
% Informasi Mengenai Dokumen
%-----------------------------------------------------------------------------%
%
% Judul laporan.
\var{\judul}{Implementing JSONField in Django}
%
% Tulis kembali judul laporan, kali ini akan diubah menjadi huruf kapital
\Var{\Judul}{Implementing JSONField in Django}
%
% Tulis kembali judul laporan namun dengan bahasa Ingris
\var{\judulInggris}{Implementing JSONField in Django}
%
% Tulis judul dalam bahasa Indonesia
\var{\judulIndonesia}{Implementasi JSONField pada Django}

%
% Tipe laporan, dapat berisi Skripsi, Tugas Akhir, Thesis, atau Disertasi
\var{\type}{Bachelor Thesis}
%
% Tulis kembali tipe laporan, kali ini akan diubah menjadi huruf kapital
\Var{\Type}{Bachelor Thesis}
%
% Tulis nama penulis
\var{\penulis}{Sage Muhammad Abdullah}
%
% Tulis kembali nama penulis, kali ini akan diubah menjadi huruf kapital
\Var{\Penulis}{Sage Muhammad Abdullah}
%
% Tulis NPM penulis
\var{\npm}{1706979455}
%
% Tuliskan Fakultas dimana penulis berada
\var{\fakultas}{Computer Science}
\Var{\Fakultas}{Computer Science}
%
% Tuliskan Program Studi yang diambil penulis
\var{\program}{Computer Science Undergraduate}
\Var{\Program}{Computer Science Undergraduate}
%
% Tuliskan Program Studi dalam bahasa Indonesia
\var{\programIndonesia}{Ilmu Komputer}
\Var{\ProgramIndonesia}{Ilmu Komputer}
%
% Tuliskan kota/kabupaten publikasi laporan
\var{\kota}{Depok}
\Var{\Kota}{Depok}
%
% Tuliskan tahun publikasi laporan
\Var{\bulanTahun}{January 2021}
%
% Tuliskan gelar yang akan diperoleh dengan menyerahkan laporan ini
\var{\gelar}{S.Kom.}
%
% Tuliskan tanggal pengesahan laporan, waktu dimana laporan diserahkan ke
% penguji/sekretariat
\var{\tanggalSiapSidang}{01 November 2020}
%
% Tuliskan tanggal keputusan sidang dikeluarkan dan penulis dinyatakan
% lulus/tidak lulus
\var{\tanggalLulus}{11 January 2021}
% Tuliskan tanggal pengesahan laporan final, waktu dimana laporan
% diserahkan ke perpustakaan
\var{\tanggalFinal}{11 January 2021}
%
% Tuliskan pembimbing
\var{\pembimbingSatu}{Dr. Ade Azurat}
\var{\pembimbingDua}{Hafiyyan Sayyid Fadhlillah, M.Kom., S.Kom.}
%
% Tuliskan penguji
\var{\pengujiSatu}{Penguji Pertama Anda}
\var{\pengujiDua}{Penguji Kedua Anda}

% Tulisan untuk menandakan halaman lampiran lanjutan
\var{\continued}{(continued)}

%-----------------------------------------------------------------------------%
% Judul Setiap Bab
%-----------------------------------------------------------------------------%
%
% Berikut ada judul-judul setiap bab.
% Silahkan diubah sesuai dengan kebutuhan.
%
\Var{\kataPengantar}{Preface}
\Var{\babSatu}{Introduction}
\Var{\babDua}{Django Web Framework and JSON}
\Var{\babTiga}{Implementation Design and Analysis}
\Var{\babEmpat}{Implementation}
\Var{\babLima}{Results and Evaluation}
\Var{\kesimpulan}{Conclusions}

% Daftar pemenggalan suku kata dan istilah dalam LaTeX
\include{hype.indonesia}
% Daftar istilah yang mungkin perlu ditandai
\input{istilah}

% Awal bagian penulisan laporan
\begin{document}

%
% Sampul Laporan
\include{sampul}
\forceclearchapter

%
% Gunakan penomeran romawi
\pagenumbering{roman}
\pagestyle{first-pages}

%
% load halaman judul dalam
\addChapter{COVER SHEET}
\include{judul_dalam}
\forceclearchapter

%
% setelah bagian ini, halaman dihitung sebagai halaman ke 2
\setcounter{page}{2}

%
% load halaman pengesahan
\addChapter{APPROVAL SHEET}
%
% Halaman Pengesahan
%
% @author  Andreas Febrian
% @version 1.01
%

\chapter*{APPROVAL SHEET}

\vspace*{0.2cm}
\noindent

\noindent
\begin{tabular}{l l p{11cm}}
	\bo{Title}&: & \judul \\
	\bo{Name}&: & \penulis \\
	\bo{NPM}&: & \npm \\
\end{tabular} \\

\vspace*{1.2cm}

\noindent This \type\ has been examined and approved.\\[0.3cm]
\begin{center}
\tanggalFinal \\[2cm]
\end{center}

\begin{center}
\begin{multicols}{2}
\underline{\pembimbingSatu}\\[0.1cm]
\type\ Advisor

\underline{\pembimbingDua}\\[0.1cm]
\type\ Advisor
\end{multicols}
\end{center}

\newpage

\forceclearchapter
%
% load halaman orisinalitas
\addChapter{ORIGINALITY STATEMENT SHEET}
%
% Halaman Orisinalitas
%
% @author  Andreas Febrian
% @version 1.01
%

\chapter*{\uppercase{ORIGINALITY STATEMENT SHEET}}
\vspace*{2cm}

\begin{center}
	\bo{This \type\ is my own work, \\
	and all sources quoted and cited
	have been specified correctly.} \\
	\vspace*{2.6cm}

	\begin{tabular}{l c l}
	\bo{Name} & : & \bo{\penulis} \\
	\bo{NPM} & : & \bo{\npm} \\
	\bo{Signature} & : & \\
	& & \\
	& & \\
	\bo{Date} & : & \bo{\tanggalSiapSidang} \\
	\end{tabular}
\end{center}

\newpage

\forceclearchapter
%
%
\addChapter{VALIDATION SHEET}
%
% Halaman Pengesahan Sidang
%
% @author  Andreas Febrian, Andre Tampubolon
% @version 1.02
%

\chapter*{VALIDATION SHEET}

\vspace*{0.4cm}
\noindent

\noindent
\begin{tabular}{@{}l}
	This \type\ is submitted by:\\
\end{tabular} \\
\begin{tabular}{@{}ll l}
	Name&: & \penulis \\
	NPM&: & \npm \\
	Program&: & \program \\
	Title of \type&: & \judulInggris \\
\end{tabular} \\

\vspace*{1.0cm}

\noindent \bo{This \type\ has been successfully defended in front of the Thesis
Committee and accepted in partial fulfillment of the requirements of the degree
\gelar\ for the \program\ Program at the Faculty of \fakultas, Universitas
Indonesia.}\\[0.2cm]

\begin{center}
	\bo{THESIS COMMITTEE}
\end{center}

\vspace*{0.3cm}

\begin{tabular}{l l l l }
	& & & \\
	Advisor 1&: & \pembimbingSatu & (\hspace*{3.0cm}) \\
	& & & \\
	Advisor 2&: & \pembimbingDua & (\hspace*{3.0cm}) \\
	& & & \\
	Examiner 1&: & \pengujiSatu & (\hspace*{3.0cm}) \\
	& & & \\
	Examiner 2&: & \pengujiDua & (\hspace*{3.0cm}) \\
\end{tabular}\\

\vspace*{2.0cm}

\begin{tabular}{ll l}
	Validated in&: & \kota \\
	Date&: & \tanggalLulus \\
\end{tabular}


\newpage

\forceclearchapter
%
%
\addChapter{STATEMENT OF APPROVAL FOR ACADEMIC PUBLICATION}
%
% @author  Andre Tampubolon, Andreas Febrian
% @version 1.01
%

\chapter*{\uppercase{
	Statement of Approval
	for Academic Publication}}

\vspace*{0.2cm}
\noindent
As a \f{civitas academica} of Universitas Indonesia, I the undersigned:
\vspace*{0.4cm}


\begin{tabular}{p{4.2cm} l p{6cm}}
	\bo{Name} & : & \penulis \\
	\bo{NPM} & : & \npm \\
	\bo{Program} & : & \program\\
	\bo{Faculty} & : & \fakultas\\
	\bo{Type of Work} & : & \type \\
\end{tabular}

\vspace*{0.6cm}
\noindent for the development of science, approve to give my work titled:
\begin{center}
	\judul
\end{center}
and its resources to Universitas Indonesia under a \bo{non-exclusive,
royalty-free license}. With the license, Universitas Indonesia has the
right to store, adapt, manage in the form of a database, take care of,
and publish my work as long as my name is included as the writer/creator
and as the copyright owner.\\

\begin{center}
	\vspace*{0.8cm}
	\begin{tabular}{lll}
		Written in&: & Depok \\
		On&: & \tanggalSiapSidang \\
	\end{tabular}\\

	\vspace*{0.5cm}
	The proponent \\
	\vspace*{1.25cm}
	(\penulis)
\end{center}

\newpage

\forceclearchapter
%
%
\addChapter{\kataPengantar}
%-----------------------------------------------------------------------------%
\chapter*{\kataPengantar}
%-----------------------------------------------------------------------------%

The basis of this thesis originally stemmed from my passion for open-source
software development. Open-source software has allowed me to tremendously
learn and grow as a computer scientist. Through open-source software, I could
study how pieces of software come together and form state-of-the-art tools that
are widely used by other people. It has opened me to various opportunities in
life that I otherwise could not achieve. Therefore, I believe that it is my
duty to contribute back to the open-source community by sharing the work I have
done on one of my favorite open-source software, the Django web framework.

Praise be to Allah, for the blessing and mercy given to me throughout my study
so that I can finally finish this thesis.

I would like to thank my thesis advisors, \pembimbingSatu\ and \pembimbingDua,
for their advice, support, and guidance throughout the writing of this thesis.

I would also like to thank my academic advisor, Siti Aminah S.Kom., M.Kom.,
and all the lecturers of the Faculty of \fakultas, Universitas Indonesia for
the support, knowledge, and wisdom they have given me during my study. The same
also goes for the teachers at SMAN 48 Jakarta, as I would not be here if it
wasn't for them.

This thesis would not be possible without the continuous support and prayers
from my family and friends. Therefore, I am very grateful for my beloved
parents, Lilis Herlinawati and Muhtadin, and my sisters, Raisha Abdillah and
Khaira Abdillah. Thanks are also due to my friends from high school and
Fasilkom UI, who have supported me since before I begin my undergraduate study.

I understand that this thesis is not perfect. Therefore, should there be any
criticisms, suggestions, and/or questions regarding this thesis, please send
them directly to me via
\href{mailto:sage.muhammad@ui.ac.id}{sage.muhammad@ui.ac.id}. I do hope that
this thesis will be useful for anyone who reads it, especially the students of
the Faculty of \fakultas, Universitas Indonesia and the Django community.

\vspace*{0.1cm}
\begin{flushright}
Depok, \tanggalSiapSidang\\[0.1cm]
\vspace*{1cm}
\penulis

\end{flushright}

%
%
\fancypagestyle{plain}{
	\fancyhead{}
	\fancyfoot[C]{\thepage}
	\fancyfoot[R]{\footnotesize \fontfamily{phv} \selectfont \bo{Universitas Indonesia}}
}
\addChapter{ABSTRAK}
%
% Halaman Abstrak
%
% @author  Andreas Febrian
% @version 1.00
%

\chapter*{Abstrak}
\singlespacing

\vspace*{0.2cm}

\noindent \begin{tabular}{l l p{10cm}}
	Nama&: & \penulis \\
	Program Studi&: & \programIndonesia \\
	Judul&: & \judulIndonesia \\
\end{tabular} \\

\vspace*{0.5cm}

\noindent
Dalam pengembangan web, penggunaan sebuah \f{web framework} merupakan praktik
yang umum untuk membangun aplikasi web. Salah satu \f{web framework} yang
populer adalah Django, sebuah \f{web framework} yang bebas dan bersumber
terbuka yang ditulis dalam bahasa Python. Di antara fitur-fitur yang tersedia
pada Django, sistem \f{object-relational mapping} (ORM) adalah yang paling
kompleks. Sistem ORM pada Django memetakan model data ke tabel dalam basis data
relasional. Model data tersebut didefinisikan sebagai \f{class} dalam Python
yang memiliki atribut yang dinamakan \f{model field}. Salah satu \f{model
field} yang tersedia pada Django adalah \verb|JSONField| yang memungkinkan
pemrogram untuk menyimpan dan mencari data semiterstruktur menggunakan format
data JSON dalam basis data relasional. Sebelum penelitian ini, \verb|JSONField|
hanya tersedia untuk sistem basis data PostgreSQL. Sementara itu, Django secara
resmi mendukung PostgreSQL, MariaDB, MySQL, SQLite, dan Oracle Database.
Penelitian ini bertujuan untuk mengimplementasikan \verb|JSONField| baru yang
kompatibel dengan seluruh sistem basis data yang didukung oleh Django.
Implementasi tersebut dibuat sebagai bagian dari program Google Summer of Code
2019 dan telah digabungkan ke dalam basis kode Django untuk perilisan Django
3.1. Selain dari implementasi tersebut, penelitian ini juga membahas contoh
penggunaan \verb|JSONField| dengan fitur validasi model yang ada pada Django
untuk memvalidasi data JSON. \\

\vspace*{0.2cm}

\noindent \bo{Kata kunci:} \\
\verb|JSONField|, JSON, Django, basis data, \f{object-relational mapping}, data
semi-terstruktur \\

\onehalfspacing
\newpage

%
%
%
% Halaman Abstract
%
% @author  Andreas Febrian
% @version 1.00
%

\chapter*{Abstract}
\singlespacing

\vspace*{0.2cm}

\noindent \begin{tabular}{@{}l l p{12.0cm}}
	Name&: & \penulis \\
	Program&: & \program \\
	Title&: & \judulInggris \\
\end{tabular} \\

\vspace*{0.5cm}

\noindent
In web development, it is a common practice to use a web framework to build web
applications. One of the most popular web frameworks is Django, a free and
open-source web framework written in Python. Among the wide range of features
in Django, the object-relational mapping (ORM) system is the most complex. The
ORM system in Django maps data models to relational database tables. The data
models are defined as Python classes that have attributes known as model
fields. One of the model fields available in Django is \verb|JSONField| that
allows programmers to store and query semi-structured data using the JSON data
format in a relational database. Before this research, \verb|JSONField| was
only available for the PostgreSQL database system. Meanwhile, Django officially
supports PostgreSQL, MariaDB, MySQL, SQLite, and Oracle Database. This research
aimed to implement a new \verb|JSONField| that is compatible with all database
systems supported by Django. In addition to the implementation, this research
also covers some examples of \verb|JSONField| usage with Django's built-in
model validation feature to validate JSON data in a \verb|JSONField|.
The process for implementing \verb|JSONField| includes researching JSON data
support on the database systems supported by Django, designing, and
implementing \verb|JSONField|. The \verb|JSONField| implementation is tested on
all database systems using automated tests. This research would hopefully
provide insights to Django users about the inner workings and validation
examples of \verb|JSONField|.\\

\vspace*{0.2cm}

\noindent \bo{Keywords:} \\
\verb|JSONField|, JSON, Django, Database, Object-relational mapping,
Semi-structured data \\

\onehalfspacing
\newpage


%
% Daftar isi, gambar, tabel, dan kode
%
\phantomsection %hack to make them clickable
\singlespacing
\tableofcontents
\onehalfspacing
\clearpage
\phantomsection %hack to make them clickable
\singlespacing
\listoffigures
\onehalfspacing
\clearpage
\phantomsection %hack to make them clickable
\singlespacing
\listoftables
\onehalfspacing
\clearpage
\phantomsection %hack to make them clickable
\addcontentsline{toc}{chapter}{\lstlistlistingname}
\singlespacing
\lstlistoflistings
\onehalfspacing

% Jika penomoran romawi selesai di ganjil
\naiveoddclearchapter
% Jika penomoran romawi selesai di genap
%\naiveevenclearchapter

%
% Gunakan penomeran Arab (1, 2, 3, ...) setelah bagian ini.
%
\pagenumbering{arabic}
\pagestyle{standard}
% \setlength{\belowcaptionskip}{+2pt}


\setoddevenheader
%-----------------------------------------------------------------------------%
\chapter{\babSatu}
%-----------------------------------------------------------------------------%

This chapter explains the background of this research and the problem to be
solved by this research.

%-----------------------------------------------------------------------------%
\section{Research Background}
%-----------------------------------------------------------------------------%

In the early days of web development, every part of a web application was coded
manually. This required extensive knowledge of how the web works. Depending on
the web application, this might include understanding of the Hypertext Transfer
Protocol (HTTP), the Hypertext Markup Language (HTML), and other things such as
databases. If not done correctly, this could lead to a lot of errors and
security holes in the web application.

Web frameworks aim to solve this problem by providing a standard way to build
web applications. Web frameworks help eliminate the overhead of implementing
common parts of web applications such as the HTTP layer, the database layer,
and the view layer. This helps web developers build web applications more
quickly and safely by focusing more on the business logic of the web
application itself instead of the more intricate details.

Among the widely-used web frameworks is Django, a free and open-source web
framework written in Python. One of the main features of Django is an
object-relational mapper (ORM) in the database layer. The ORM maps the data
models, represented as Python classes, to tables in a relational database. The
columns of a table are defined as attributes (called fields) inside the model
class.

While relational databases are useful in handling structured data, there has
been an increase in the popularity of non-relational databases, commonly known
as NoSQL databases. NoSQL encourages the simplicity of database design, which
makes the process of scaling the database to multiple clusters of machines
easier. NoSQL databases are also considered as more flexible due to its dynamic
schema.

However, NoSQL databases also have their own downsides. Most NoSQL databases
disregard the Atomicity, Consistency, Isolation, and Durability (ACID)
principles commonly found in relational database transactions. They also lack
the ability to perform joins across tables, which makes it harder to deal with
entity relations.

Nowadays, some relational database systems have come up with their own
solutions of providing a hybrid data model. This is done by combining
structured data with some form of semi-structured data, most commonly in
JavaScript Object Notation (JSON) format. This gives all the benefits of using
a relational database, while still allowing the simplicity and dynamic nature
of semi-structured data.

From version 1.9 until version 3.0, Django had provided an implementation of
JSONField. However, it was exclusively available for PostgreSQL. This JSONField
allowed its users to store and query JSON data in a relational database column
using the jsonb data type found only on PostgreSQL at the time.

Django officially supports PostgreSQL, SQLite, MySQL, MariaDB, and Oracle
Database backends. Over the years since the initial PostgreSQL JSONField
implementation, those database systems have developed their own support for
dealing with JSON data. However, Django had yet to implement JSONField for
database backends other than PostgreSQL.

The limited support for JSONField prompted the Django community to create their
own JSONField implementations for other database backends. Some of them target
one specific database backend and utilize various functions provided by the
database system to add extended querying capabilities. Some other
implementations focus on cross-database support, so they are only implemented
as text-based fields that do not have any extended querying capabilities.

The individual efforts for implementing JSONField results in the abundance of
third-party JSONField packages on the Python Package Index. Some of them are
very popular, such as the jsonfield package that has more than 1100 stars on
GitHub. This indicates a popular demand for JSONField. However, it also
indicates a fragmentation problem as people use different JSONField packages.
Thus, there's a motivation to bring JSONField into Django's core model fields,
with cross-database and extended querying support in mind, as well as
compatibility with the previous implementation of JSONField.

%-----------------------------------------------------------------------------%
\section{Permasalahan}
%-----------------------------------------------------------------------------%
\todo{Sebutkan permasalahan penelitian Anda dari latar belakang tersebut.}

%-----------------------------------------------------------------------------%
\subsection{Definisi Permasalahan}
%-----------------------------------------------------------------------------%
Berikut ini adalah rumusan permasalahan dari penelitian yang dilakukan:
\begin{itemize}
	\item Bagaimana cara membuat pertanyaan penelitian?
\end{itemize}
\todo{Tuliskan permasalahan yang ingin diselesaikan. Bisa juga berbentuk pertanyaan}


%-----------------------------------------------------------------------------%
\subsection{Batasan Permasalahan}
%-----------------------------------------------------------------------------%
Berikut ini adalah asumsi yang digunakan sebagai batasan penelitian ini:
\begin{itemize}
	\item Salah satu batasannya adalah, ini hanya \f{template}.
\end{itemize}
\todo{Umumnya ada asumsi atau batasan yang digunakan untuk menjawab
pertanyaan-pertanyaan penelitian diatas.}


%-----------------------------------------------------------------------------%
\section{Tujuan Penelitian}
%-----------------------------------------------------------------------------%
Berikut ini adalah tujuan penelitian yang dilakukan:
\begin{itemize}
	\item Untuk memberikan \f{template} yang dapat mempermudah skripsi orang
	lain.
\end{itemize}
\todo{Tuliskan tujuan penelitian Anda di bagian ini.}


%-----------------------------------------------------------------------------%
\section{Posisi Penelitian}
%-----------------------------------------------------------------------------%
\todo{Sebutkan posisi penelitian Anda. Ada baiknya jika Anda menggunakan gambar
atau diagram. Template ini telah menyediakan contoh cara memasukkan gambar.}

\begin{figure}
	\centering
	\includegraphics[width=0.4\textwidth]{pics/makara.png}
	\caption{Penjelasan singkat terkait gambar.}
	\label{fig:research_position}
\end{figure}

\todo{Jelaskan \pic~\ref{fig:research_position} di sini.}


%-----------------------------------------------------------------------------%
\section{Langkah Penelitian}
%-----------------------------------------------------------------------------%
Berikut ini adalah langkah penelitian yang telah dilakukan:
\begin{enumerate}
	\item Tinjauan literatur \\
	Pada tahap ini, dipelajari teori-teori yang terkait dengan penelitian ini
	untuk mendapatkan konsep dasar yang dibutuhkan dalam mencapai tujuan
	penelitian.
	\item Analisis implementasi dan kesimpulan \\
	Pada tahap ini, digunakan studi kasus untuk analisis terkait kegunaan
	\f{template}. Setelah melakukan analisis tersebut, ditarik kesimpulan
	keseluruhan dari penelitian ini.
\end{enumerate}


%-----------------------------------------------------------------------------%
\section{Sistematika Penulisan}
%-----------------------------------------------------------------------------%
Sistematika penulisan laporan adalah sebagai berikut:
\begin{itemize}
	\item Bab 1 \babSatu \\
	    Bab ini mencakup latar belakang, cakupan penelitian, dan pendefinisian
	    masalah.
	\item Bab 2 \babDua \\
	    Bab ini mencakup pemaparan terminologi dan teori yang terkait dengan
	    penelitian berdasarkan hasil tinjauan pustaka yang telah digunakan,
	    sekaligus memperlihatkan kaitan teori dengan penelitian.
	\item Bab 3 \babTiga \\
	    Apa itu Bab 3?
	\item Bab 4 \babEmpat \\
		Apa itu Bab 4?
	\item Bab 5 \babLima \\
	    Apa itu Bab 5?
	\item Bab 6 \kesimpulan \\
	    Bab ini mencakup kesimpulan akhir penelitian dan saran untuk
	    pengembangan berikutnya.
\end{itemize}

\todo{Anda bisa mengubah atau menambahkan penjelasan singkat mengenai isi
masing-masing bab. Setiap tugas akhir pasti ada yang berbeda pada bagian ini.}

\clearchapter
%-----------------------------------------------------------------------------%
\chapter{\babDua}
%-----------------------------------------------------------------------------%

This chapter focuses on building a conceptual basis by reviewing existing
literature and documentation of the underlying concepts of this research.

%-----------------------------------------------------------------------------%
\section{Web Framework}
%-----------------------------------------------------------------------------%

A web framework is a software framework that is designed to support the
development of web applications. It consists of reusable components that solve
common problems in web application development. Most web frameworks provide
safe defaults to prevent security problems in the web application. This
convenience helps programmers build web applications faster and safer. Some web
frameworks also help maintain better application structure by enforcing
software design patterns.

The common components of a web framework are the dispatcher, the decoder, the
generator, and the store \cite{schwarz_webframework}. The dispatcher maps
Uniform Resource Locators (URLs) to application code that is invoked when an
HTTP request is received. The decoder decodes the client request, which allows
the web application to read parameters, headers, and request body data sent as
part of the request. The generator constructs the output that is sent as a
response to the client. The store holds inter-request data, which is usually
stored in a database system.

%-----------------------------------------------------------------------------%
\section{Django and Object-Relational Mapping}
%-----------------------------------------------------------------------------%

Django is a free and open-source web framework written in Python \cite{django}.
Its overall philosophies are loose coupling, less code, quick development,
don't repeat yourself (DRY), explicit is better than implicit, and consistency
\cite{django:philosophies}. Django follows what it calls the
model-template-view architectural pattern, which shares similarities to the
model-view-controller pattern \cite{django:faq}. It consists of an
object-relational mapping (ORM) system that handles the interaction with
database systems, a template system that defines how the data looks like to the
user, and a view system that defines which data is presented to the user.

Object-relational mapping (ORM) is a programming technique that allows its
users to query and manipulate data stored in a relational database using an
object-oriented paradigm \cite{linskey:orm}. Object query, load, and delete
operations are translated into SQL SELECT and DELETE statements, while object
allocation and modification operations are translated into SQL INSERT and
UPDATE statements. To achieve this, the scalar values retrieved from the
database are converted into object instances in the program and vice versa.

The ORM system in Django maps data models to relational database tables. The
data models are represented as Python classes. The model class also has
attributes, called model fields, which represent the columns of the database
table. These model fields define the data types that are used in the database.
For example, the CharField model field uses the VARCHAR data type in the
database. In addition, these model fields also define the behavior of the
columns, such as their NOT NULL and UNIQUE constraints in the database.

Aside from storing and loading data, the ORM system in Django also provides
querying capabilities through the use of lookups and transforms. Lookups and
transforms are specified using keyword arguments in the form of
\code{field\_\_lookuptype=value}, where \code{lookuptype} can be a lookup or
transform name. Lookups define how the WHERE clause of an SQL query is composed
by Django. Transforms are used to transform a model field from one form to
another (e.g. extracting the year from a DateField). For example, a query
specified in Python as shown in \autoref{code:query1py} will be translated into
an SQL query equivalent shown in \autoref{code:query1sql}.

\lstinputlisting[
	language=Python,
	caption={
		A query on a DateField named \code{pub\_date}
		with the \code{year} transform and \code{gte} lookup.},
	label=code:query1py]{codes/2-query1.py}

\lstinputlisting[
	language=SQL,
	caption={An SQL query equivalent to \autoref{code:query1py}.},
	label=code:query1sql]{codes/2-query1.sql}

Django officially supports five database backends: PostgreSQL, MariaDB, MySQL,
SQLite, and Oracle Database \cite{django:databases}. Each database backend
(other than SQLite) requires a compatible database driver to be installed. The
database backends provided by Django adapt the database drivers so that they
can be used with the ORM system. These backends are implemented in the
django.db.backends module.

%-----------------------------------------------------------------------------%
\section{JSON on Relational Database Systems Supported by Django}
%-----------------------------------------------------------------------------%

JavaScript Object Notation (JSON) is a lightweight, text-based data-interchange
format \cite{json}. It is designed to be human-readable and easy for machines
to parse and generate. It is based on a subset of the JavaScript standard, but
it is completely language independent. It uses conventions that are similar to
popular programming languages such as C, C++, C\#, Java, JavaScript, Python,
and many others. These properties make JSON a widely-used data-interchange
format, especially in web applications.

JSON is built on two structures: an unordered set of key-value pairs (called
JSON objects) and an ordered list of values (called JSON arrays). The values
can be scalar values such as strings, numbers, booleans, or null. However, they
can also be JSON objects or JSON arrays, which means that they can be nested
to form more complex data structures. An example of a JSON object that contains
a JSON array and other JSON objects can be seen in \autoref{code:json}.

\lstinputlisting[
	language=Python,
	caption=A JSON object that contains a JSON array and other JSON objects.,
	label=code:json]{codes/2-json.json}

To utilize the flexibility of JSON, some relational databases have implemented
support for storing and querying JSON data. This support is commonly
implemented by providing database functions to operate on JSON data stored as
text or, in some cases, a native JSON data type. JSON support on the database
systems supported by Django varies between one another. This section explains
how those database systems provide JSON support.

\subsection{PostgreSQL}

PostgreSQL supports a native JSON data type and JSON functions as of version
9.2 \cite{postgresql:9.2}. Data stored using the JSON data type is stored as
text. However, it has the advantage of validating that each stored value is
valid JSON. The two JSON functions included in the 9.2 release allow users to
convert a row or array in the database into JSON.

As of version 9.4, PostgreSQL also supports a JSONB data type and more JSON
functions \cite{postgresql:9.4}. JSONB supports fast data retrievals and simple
expression search queries using Generalized Inverted Indexes (GIN). Data stored
using the JSONB data type is stored as binary. This makes storing data with the
JSONB data type slightly slower compared to the JSON data type as it needs to
be encoded into binary prior to saving. However, it is significantly faster to
process, since the processing functions do not need to reparse the data on each
execution. The JSON functions introduced in the 9.4 release enable users to
extract and manipulate JSON data with a performance that matches or surpasses
document-based databases.

\subsection{MariaDB and MySQL}

MariaDB is highly compatible with MySQL, albeit with some behavior differences
\cite{mariadb:compatibility}. This is because MariaDB is originally designed to
be a completely open-source drop-in replacement for MySQL. MariaDB's client
protocol is binary compatible with MySQL's client protocol, therefore Django
uses the same database backend for MariaDB and MySQL \cite{django:databases}.
Both database systems have compatible JSON functions to operate on JSON data,
but the data is stored differently.

MySQL supports a native JSON data type as of version 5.7.8 \cite{mysql:json}.
The JSON data type provides automatic validation of JSON data. It is internally
stored as binary, which permits quick read access as the value does not need to
be parsed from a text representation. Data stored using the JSON data type can
be indexed by using a generated column that extracts a scalar value from the
JSON data.

MariaDB has a JSON data type as of version 10.2.7 \cite{mariadb:json}. The JSON
data type was introduced for compatibility reasons with MySQL's JSON data type.
However, the JSON data type is just an alias for the LONGTEXT data type.
Despite the data is stored as a string rather than binary, it can be validated
using the provided JSON\_VALID function as a CHECK constraint.

\subsection{SQLite}

SQLite has the JSON1 extension as of version 3.9.0 \cite{sqlite:3.9.0}. The
JSON1 extension is a loadable extension that includes functions that can be
used to manage JSON content \cite{sqlite:json1}. As of this writing, the JSON1
extension stores JSON data as text. The JSON1 extension includes a JSON\_VALID
function that can be used as a CHECK constraint for JSON data.

\subsection{Oracle Database}

Oracle Database supports storing, validating, querying, and indexing JSON data
as of version 12.1.0.2 \cite{oracle:12.1.0.2}. JSON data is stored using the
VARCHAR2, CLOB, and BLOB data types \cite{oracle:json}. Unlike the other
database systems, Oracle Database does not support storing scalar values in
JSON columns, which means that the top-level values in JSON columns are either
JSON objects or JSON arrays. JSON validation can be enforced using the IS JSON
condition as a CHECK constraint. Querying JSON data can be performed by
utilizing the provided JSON functions. Indexing JSON data can be implemented
using bitmap, function-based, or composite B-tree indexes.

%-----------------------------------------------------------------------------%
\section{Keterkaitan Teori Dengan Penelitian}
%-----------------------------------------------------------------------------%
\todo{Ada baiknya setelah menjelaskan teori-teori, Anda menjelaskan apa kaitan teori tersebut dengan penelitian Anda. Hal ini tentunya membantu pembaca dalam memahami bahwa teori yang Anda paparkan memang penting untuk memahami penelitian Anda nantinya.}

\begin{figure}
	\centering
	\includegraphics[width=\textwidth]{pics/research_concept_map.png}
	\caption{Keterkaitan konsep hasil studi literatur terhadap penelitian}
	\label{fig:research_concept_map}
\end{figure}

\todo{Jelaskan \pic~\ref{fig:research_concept_map} di sini. Setiap gambar pada
tugas akhir butuh penjelasan. Gambar hadir untuk mempermudah membaca memahami
konteks, tetapi tidak bisa berdiri sendiri tanpa penjelasan. Terkait gambar,
Anda juga bisa mengatur skalanya. Gambar kali ini lebarnya 0,8x dari lebar teks
halaman.}

\clearchapter
%-----------------------------------------------------------------------------%
\chapter{\babTiga}
%-----------------------------------------------------------------------------%

This chapter explains the design and analysis of the \code{JSONField}
implementation, as well as \code{JSONField} data validation examples.

%-----------------------------------------------------------------------------%
\section{\code{JSONField}}
%-----------------------------------------------------------------------------%

There are two kinds of \code{JSONField}: the model field and the form field.
The model field is used as an abstraction of the JSON data in the database,
which lets its users manipulate JSON data in the form of Python objects.
The form field is used for accepting JSON data in forms, such as a Django
\code{ModelForm}. Both fields can be validated using Django's built-in
validation feature with custom-made validator functions.

The model field holds JSON data that can be stored to and retrieved from the
database. In Python, the data is represented in Python's built-in formats:
dictionaries, lists, strings, numbers, booleans, and \code{None}. When saving
the model, Django executes an SQL \code{INSERT} or \code{UPDATE} query to the
database. In order to pass the JSON data in the SQL query, the data has to be
serialized into a JSON-encoded string. When retrieving the model instance,
Django executes an SQL \code{SELECT} query to the database. The data is
retrieved as a JSON-encoded string, which needs to be deserialized, or decoded,
into a Python object.

Python has a built-in \code{json} library that can be used to encode/decode
Python objects into/from JSON-encoded strings \cite{python:json}. The encoding
and decoding functionalities are provided through the \code{json.dumps()} and
\code{json.loads()} functions, respectively, as shown in
\autoref{fig:encodecode}. By default, the \code{json.dumps()} function uses the
\code{json.JSONEncoder} class, while the \code{json.loads()} function uses the
\code{json.JSONDecoder} class. These classes support the Python \code{dict},
\code{list}, \code{str}, \code{int}, \code{float}, \code{True}, \code{False},
and \code{None} data types and objects. To support other data types and
objects, customized \code{json.JSONEncoder} and \code{json.JSONDecoder}
subclasses can be used for the functions by supplying the subclass as the
\code{cls} argument.

\begin{figure}
	\centering
    \includegraphics[width=0.66\textwidth]{pics/encodecode.png}
	\caption{The use of the \code{json.dumps()} and \code{json.loads()}
	functions to encode a Python object and decode a JSON-encoded string,
	respectively.}
	\label{fig:encodecode}
\end{figure}

The form field utilizes the \code{json} library for serialization and
deserialization when handling JSON input from the client. If the data being
deserialized is not a valid JSON document, the \code{json.loads()} function
will raise a \code{JSONDecodeError}. This error is used by Django to provide
a basic validation functionality by catching the error and raising a
\code{ValidationError} instead.

\begin{table}
	\centering
	\texttt{
\begin{tabular}{|c|c|c|c|}
\hline
\no{Python}    & \no{JSON} & \no{SQL}            & \no{JSON-encoded string} \\ \hline
'' \no{or} ""  & ""        & ''                  & '""'                     \\ \hline
\{\}           & \{\}      & \no{not applicable} & '\{\}'                   \\ \hline
[]             & []        & \no{not applicable} & '[]'                     \\ \hline
None           & null      & NULL                & 'null'                   \\ \hline
\end{tabular}
}
	\caption{Comparison of empty values and their equivalents in Python, JSON, and SQL.
	The "JSON-encoded string" column shows the Python values after serialization
	with \code{json.dumps()}.}
	\label{table:emptyvalues}
\end{table}

When serializing and deserializing JSON data for the model field, the Python
\code{None} object, the JSON \code{null} value, and the SQL \code{NULL} value
should be taken into consideration. The model field uses JSON-encoded strings
to store JSON values, as there are no SQL equivalents for JSON objects and
arrays. Following the patterns of other values shown in
\autoref{table:emptyvalues}, the \code{None} object should be stored as the
JSON-encoded \code{'null'} string on the database. However, according to the
Python DB-API 2.0 specification (which Django follows), the \code{None} object
is reserved for the SQL \code{NULL} value for both input and output
\cite{db-api2}. Therefore, the model field should skip the serialization and
deserialization for \code{None} (if it's stored as the top-level value) and let
the database driver handle it as SQL \code{NULL}.

In order to store and query the JSON \code{null} value as the top-level value
of \code{JSONField}, the \code{Value} class from the \code{django.db.models}
module can be utilized. A \code{Value()} object wraps a literal SQL value to be
used in an SQL expression \cite{django:value}. This literal SQL value is not
processed by Django. In the case of \code{JSONField}, that means the value is
not passed into the \code{json.dumps()} function. Therefore, the JSON
\code{null} value can be stored as a JSON-encoded SQL string literal (i.e.
\code{Value('null')}), which will be stored as \code{'null'} and not
\code{'"null"'}. However, when the value is retrieved from the database as
\code{'null'}, Django will process the value using \code{json.loads()}, which
will return \code{None}. Meanwhile, when \code{None} is saved back to the
database, it will be saved as SQL \code{NULL}. This behavior may cause
confusion as discussed on the GitHub pull request shown in
\autoref{fig:nulldiscussion}.\footnote{\url{
	https://github.com/django/django/pull/11452\#discussion\_r335375254}}
To avoid this confusion, Django does not recommend working with JSON
\code{null} as the top-level value of JSONField.

\begin{figure}
	\centering
    \includegraphics[width=1.0\textwidth]{pics/null_discussion.png}
	\caption{A brief discussion regarding the use of JSON \code{null} as the
	top-level value of a \code{JSONField}.}
	\label{fig:nulldiscussion}
\end{figure}

As with string-based fields such as \code{CharField} and \code{TextField},
Django recommends avoiding the use of SQL \code{NULL} for \code{JSONField}. If
the SQL \code{NULL} is used on a string-based field, it means that there are
two possible values for "no data": \code{NULL} and the empty string
(\code{''}). On \code{JSONField}, it means that there are the empty JSON object
(\code{'\{\}'}), the empty JSON array (\code{'[]'}), the empty JSON string
(\code{'""'}), the JSON \code{null} value (\code{'null'}), and the SQL
\code{NULL}. To avoid redundancy, Django recommends setting \code{null=False}
(to enforce database-level \code{NOT NULL} constraint) and providing a suitable
default for empty values, such as \code{default=dict}.

%-----------------------------------------------------------------------------%
\section{\code{JSONField} Lookups and Transforms}
%-----------------------------------------------------------------------------%

To query JSON data, the previous implementation of \code{JSONField} included
\code{JSONField}-specific lookups and transforms. The lookups and transforms
consisted of containment lookups, key existence lookups, and path transforms.
These lookups and transforms were implemented by using JSON operators that are
only available on PostgreSQL. To implement them on other database systems,
those operators need to be substituted with their JSON function equivalents.

\noindent
\begin{minipage}{\linewidth}
\lstinputlisting[language=Python, caption={A demonstration of how the
\code{contains} and \code{contained\_by} lookups are used.},
label=code:containment]{codes/3-containment.py}
\end{minipage}

The containment lookups consist of the \code{contains} and \code{contained\_by}
lookups. The \code{contains} lookup is overridden on \code{JSONField}.
Normally, it is used to query for strings using a case-sensitive substring
containment test. On \code{JSONField}, it is used to query for JSON data using
a subset containment test. The query will return objects with JSON data that
contains the lookup value as a subset. Meanwhile, the \code{contained\_by}
lookup is the inverse of the \code{contains} lookup, which means that it looks
for objects with JSON data that is contained by the lookup value. The
\code{contains} and \code{contained\_by} lookups are demonstrated in
\autoref{code:containment}.

The containment lookups can be implemented using JSON operators or JSON
functions provided by some of the database systems. On PostgreSQL, the
\code{contains} and \code{contained\_by} lookups can be implemented using the
\code{@>} and \code{<@} operators, respectively. On MariaDB and MySQL, both
lookups can be implemented using the \code{JSON\_CONTAINS} function by
switching the first and second arguments for the \code{contained\_by} lookup.
Unfortunately, SQLite and Oracle Database do not have a similar function, so
these lookups are left unsupported on both databases.

% Key Existence Lookups

% Key, Index, and Path Transforms

\clearchapter
%-----------------------------------------------------------------------------%
\chapter{\babEmpat}
%-----------------------------------------------------------------------------%

This chapter explains the implementation of \code{JSONField} and
\code{JSONField}-specific lookups and transforms that can be used on all
database backends supported by Django.

%-----------------------------------------------------------------------------%
\section{Database Backend Adjustments}
%-----------------------------------------------------------------------------%

Before implementing the \verb|JSONField| class, it is important to remember
that there are differences between the database systems in regards to handling
JSON data. For example, each database system uses a different data type to
store JSON data. In addition, the database driver (e.g. Psycopg 2 for
PostgreSQL) may automatically deserialize JSON data into Python objects. These
differences need to be taken into account when implementing \verb|JSONField|.

Django keeps track of the database systems' differences in the
\verb|django.db.backends| module. Each database system has its own database
backend submodule.\footnote{Except for MariaDB and MySQL, as they can share the
same driver and backend.} The database backends' classes are extended from the
classes in the \verb|django.db.backends.base| submodule. For example, the
\verb|DatabaseFeatures| class stores boolean flags and other feature-related
information as the class' attributes, and the default values for these
attributes are defined in the \verb|BaseDatabaseFeatures| class.

\listing
{Python}
{The new flags for \code{JSONField} in the \code{BaseDatabaseFeatures} class.}
{code:basedbfeatures}
{codes/4-basedbfeatures.py}

To implement the \verb|JSONField| class, three flags are added to the
\verb|BaseDatabaseFeatures| class. The first flag is
\verb|supports_json_field|, which shows whether the database system supports
\verb|JSONField|. The second flag is \verb|supports_primitives_in_json_field|,
which is used to determine whether the database system supports storing JSON
scalar values (strings, numbers, booleans, and \verb|null|) directly in a
\verb|JSONField|.\footnote{This flag is not used in the \code{JSONField}
implementation itself, but it is useful to store this information to be used in
tests.} The third flag is \verb|has_native_json_field|, which shows whether the
database system has a native JSON data type and the database driver
automatically deserializes JSON data into Python objects. These flags are
overridden in each database backend's \verb|DatabaseFeatures| class as
necessary.

\listing
[0 pt]
{Python}
{The \code{DatabaseWrapper} class of the PostgreSQL backend.}
{code:backends-postgresql-1}
{codes/4-backends-postgresql-1.py}

For the PostgreSQL database backend, the \verb|DatabaseWrapper| class is
modified to add support for \verb|JSONField|. The \verb|data_types| property of
the class is updated to map \verb|JSONField| to the \verb|jsonb| data type, as
shown by line 7 of the listing. The \verb|jsonb| data type automatically checks
whether the data is valid JSON, so there is no need for a \verb|CHECK|
constraint.

\listing
{Python}
{The \code{DatabaseFeatures} class of the PostgreSQL backend.}
{code:backends-postgresql-2}
{codes/4-backends-postgresql-2.py}

PostgreSQL has a native JSON data type and the driver (Psycopg 2) automatically
deserializes JSON data into Python objects. Thus, the
\verb|has_native_json_field| flag in the \verb|DatabaseFeatures| class of the
backend is set to \verb|True|, as shown by line 4 of
\autoref{code:backends-postgresql-2}. The other flags
(\verb|supports_json_field|\footnote{Django supports PostgreSQL 9.5 and higher,
while \code{jsonb} is available as of version 9.4, so \code{JSONField} is
always supported.} and \verb|supports_primitives_in_json_field|) are not
overridden, as their values from \verb|BaseDatabaseFeatures| are already
correct.

\listing
{Python}
{The \code{DatabaseOperations} class of the PostgreSQL backend.}
{code:backends-postgresql-3}
{codes/4-backends-postgresql-3.py}

To allow a custom decoder for \verb|JSONField| on PostgreSQL, Psycopg 2's
automatic deserialization of JSON data should be disabled. Disabling the
feature can be done by casting the data to \verb|text| or registering a stub
function for Psycopg 2's \verb|jsonb| converter
\cite{psycopg2:json-adaptation}. The casting option was chosen during this
research, as registering a stub function may break compatibility with the
previous implementation that relied on the automatic
deserialization.\footnote{The casting operation is efficient and does not
involve a copy, so this decision should not significantly affect the
performance of \code{JSONField} \cite{psycopg2:json-adaptation}.} The casting
operation is stored as a method in the \verb|DatabaseOperations| class of the
backend. In the case of PostgreSQL, casting data to \verb|text| can be done
using the \verb|::text| syntax, as shown by line 6 of
\autoref{code:backends-postgresql-3}.

\listing
{Python}
{The \code{DatabaseWrapper} class of the MySQL backend.}
{code:backends-mysql-1}
{codes/4-backends-mysql-1.py}

The MySQL database backend, which is used for MariaDB and MySQL, is modified to
add support for \verb|JSONField|. For this backend, the \verb|DatabaseWrapper|
class is modified by adding a mapping from \verb|JSONField| to the \verb|json|
data type in the \verb|data_types| property, as shown by line 6 of
\autoref{code:backends-mysql-1}. The \verb|data_type_check_constraints|
property method is also modified to add the \verb|JSON_VALID()| function as a
\verb|CHECK| constraint for MariaDB prior to version 10.4.3, as shown by lines
15 and 16 of the listing. Later versions of MariaDB have the \verb|CHECK|
constraint automatically enabled when using the \verb|JSON| data type alias,
while MySQL automatically validates the JSON syntax when using the \verb|JSON|
data type. Therefore, there is no need to modify
\verb|data_type_check_constraints| for these systems.

\listing
{Python}
{The \code{DatabaseFeatures} class of the MySQL backend.}
{code:backends-mysql-2}
{codes/4-backends-mysql-2.py}

Meanwhile, the \verb|DatabaseFeatures| class is modified by turning
\verb|supports_json_field| into a cached property method. The method returns
\verb|True| for MariaDB version 10.2.7 or greater and MySQL version 5.7.8 or
greater (and \verb|False| otherwise), as shown by lines 4 to 8 of
\autoref{code:backends-mysql-2}. The other flags
(\verb|supports_primitives_in_json_field| and \verb|has_native_json_field|)
use the values from \verb|BaseDatabaseFeatures| as they are already correct.

\listing
{Python}
{The \code{DatabaseWrapper} class of the SQLite backend.}
{code:backends-sqlite3-1}
{codes/4-backends-sqlite3-1.py}

The \verb|DatabaseWrapper| class of the SQLite backend is modified to add the
mapping from \verb|JSONField| to \verb|text| in the \verb|data_types| property,
as shown by line 7 of \autoref{code:backends-sqlite3-1}. It is also modified to
apply the \verb|JSON_VALID()| check constraint to the database column. As the
function returns \verb|0| (false) for SQL \verb|NULL| values, an
\verb|OR "%(column)s" IS NULL| clause is added to support storing SQL
\verb|NULL| values, as shown by line 12 of the listing.

\listing
{Python}
{The \code{DatabaseFeatures} class of the SQLite backend.}
{code:backends-sqlite3-2}
{codes/4-backends-sqlite3-2.py}

Meanwhile, \verb|DatabaseFeatures| class is modified to check for JSON support
on SQLite. The \verb|supports_json_field| flag is turned into a cached property
method and the value is determined by trying to execute a \verb|SELECT| query
with the \verb|JSON()| function included in the JSON1 extension, as shown by
lines 4 to 8 of \autoref{code:backends-sqlite3-2}. If the query does not throw
an exception, it means the JSON1 extension is enabled and the flag is set to
\verb|True| (and \verb|False| otherwise), as shown by lines 9 to 11 of the
listing. The support check is done this way because SQLite does not provide a
direct way to check whether the JSON1 extension is loaded. The other flags
(\verb|supports_primitives_in_json_field| and \verb|has_native_json_field|)
are derived from \verb|BaseDatabaseFeatures|.

\listing
{Python}
{The \code{DatabaseWrapper} class of the Oracle Database backend.}
{code:backends-oracle-1}
{codes/4-backends-oracle-1.py}

For the Oracle Database backend, the \verb|DatabaseWrapper| class is modified
to add the data type mapping and \verb|CHECK| constraint. The \verb|data_types|
property is updated to map \verb|JSONField| to \verb|NCLOB|, as shown by line 7
of \autoref{code:backends-oracle-1}. The \code{NCLOB} data type is chosen to
allow data larger than 32,767 bytes and to have Unicode support
\cite{oracle:overview-json, oracle:database-concepts}. The
\verb|data_type_check_constraints| property is also updated to apply the
\verb|IS JSON| constraint for \verb|JSONField|, as shown by line 12 of the
listing.

\listing
{Python}
{The \code{DatabaseOperations} class of the Oracle Database backend.}
{code:backends-oracle-2}
{codes/4-backends-oracle-2.py}

The cx\_Oracle database driver returns \verb|CLOB| and \verb|BLOB| values from
the database as \verb|LOB| objects in Python \cite{cxoracle:lob}. Meanwhile,
the \verb|LOB| objects need to be converted to Python strings in order to
perform deserialization. To do this, the \verb|get_db_converters()| method in
the \verb|DatabaseOperations| class is overridden so that \verb|JSONField| uses
the \verb|convert_textfield_value| converter, as shown by lines 7 and 8 of
\autoref{code:backends-oracle-2}.\footnote{\code{TextField} also uses the
\code{NCLOB} data type on Oracle Database, so its converter can be reused for
\code{JSONField}.}

\listing
{Python}
{The \code{DatabaseFeatures} class of the Oracle Database backend.}
{code:backends-oracle-3}
{codes/4-backends-oracle-3.py}

As for the \verb|DatabaseFeatures| class, the \verb|supports_primitives_in_json_field|
flag is overridden to \verb|False|, as shown by line 4 of
\autoref{code:backends-oracle-2}. This flag reflects the behavior of Oracle
Database that only allows storing JSON objects and arrays in a column with the
\verb|IS JSON| constraint enabled. The other flags
(\verb|supports_json_field|\footnote{Oracle Database introduced JSON support in
version 12.1.0.2 and Django supports Oracle Database version 12.2 and higher,
so \code{JSONField} is always supported.} and \verb|has_native_json_field|) are
derived from \verb|BaseDatabaseFeatures|.

All of the database backends have now been adjusted with the necessary changes.
The changes were mostly made for mapping \verb|JSONField| to the correct data
type on the database, as well as the \verb|CHECK| constraints to ensure that
the data is valid JSON. With these changes in place, the \verb|JSONField| class
can now be implemented.

%-----------------------------------------------------------------------------%
\section{\code{JSONField}}
%-----------------------------------------------------------------------------%

In the Django codebase, model fields are defined in the
\verb|django.db.models.fields| module as subclasses of \verb|Field|, but they
can be imported from the \verb|django.db.models| module for convenience. Model
fields that are relatively simple are implemented in the \verb|__init__.py|
file of the module. For model fields that require more complex implementation
(e.g. file fields and relational fields), they are defined in their own files
in the \verb|fields| directory. Due to the complexity of \verb|JSONField| and
its extended querying capabilities, the implementation is defined in the
\verb|json.py| file in the \verb|fields| directory.

\listing
{Python}
{The \code{deconstruct()} method of \code{Field}.}
{code:field-deconstruct}
{codes/4-field-deconstruct.py}

Model fields have a \verb|deconstruct| method that defines how the field can be
reconstructed from a database migration. The method returns a 4-tuple that
consists of: the name of the field on the model, the import path of the field,
a list of positional arguments, and a dictionary of keyword arguments
\cite{django:model_fields}. By convention, model fields that are defined in
their own files like \verb|JSONField| should still be imported from
\verb|django.db.models|. Thus, the \verb|deconstruct()| method of the
\verb|Field| class is modified to shorten the path for \verb|JSONField|'s
module, as shown by lines 15 to 17 of \autoref{code:field-deconstruct}.

\listing
{Python}
{The \code{check()} method of \code{JSONField} model field.}
{code:jsonfield-check}
{codes/4-jsonfield-check.py}

The \verb|JSONField| class implements some checks to make sure that it is
used correctly. In addition to the checks included from the \verb|Field| class,
\verb|JSONField| also extends the \verb|CheckFieldDefaultMixin| class, as shown
by line 1 of \autoref{code:jsonfield-check}. The mixin adds a check to ensure
that any default value set for the field should be a callable instead of an
instance, so that it's not shared between all instances of the field. The
\verb|JSONField| class also overrides the \verb|check()| method to include the
\verb|_check_supported()| method in the checking process, as shown by lines 4
to 8 of the listing.

\listing
{Python}
{The \code{\_check\_supported()} method of \code{JSONField} model field.}
{code:jsonfield-checksupported}
{codes/4-jsonfield-checksupported.py}

The \verb|_check_supported()| method checks all of the connected databases for
\verb|JSONField| support. If migrations for the model (that the field is
attached to) are not allowed on the database by the database router, then the
database is skipped from the check, as shown by lines 4 and 5 of
\autoref{code:jsonfield-checksupported}. Otherwise, the
\verb|supports_json_field| feature flag is checked. Lines 7 to 17 show that if
the flag's value is \verb|False| and the model is not explicitly defined to
require the feature flag, then a system error is added to the lists of errors
to be returned by the \verb|check| method and shown to the programmer.

\listing
{Python}
{The \code{get\_internal\_type()} method of \code{JSONField} model field.}
{code:jsonfield-getinternaltype}
{codes/4-jsonfield-getinternaltype.py}

To determine the key that is used to get the data type and check constraints
from the database backends, the \verb|get_internal_type()| method is used. By
default, the method returns the name of the class (using
\verb|self.__class__.__name__|), so it is not necessary to override it.
However, other model fields override the method to explicitly return the name
of the class, so that other classes may extend the model fields without having
to override the \verb|get_internal_type()| method unless it is needed. Thus,
the \verb|get_internal_type()| method is overridden to explicitly return the
string \verb|'JSONField'|, as shown by line 2 of
\autoref{code:jsonfield-getinternaltype}.

\listing
{Python}
{The \code{get\_prep\_value()} method of \code{JSONField} model field.}
{code:jsonfield-ser}
{codes/4-jsonfield-ser.py}

In the previous chapter, \verb|JSONField| is designed to serialize Python
objects into JSON-encoded strings by utilizing the built-in \verb|json| library
in Python. The serialization process is needed in order to store Python objects
as JSON in the database. To convert Python objects to query values, the
\verb|get_prep_value()| method should be overridden. The serialization is shown
by line 4 of \autoref{code:jsonfield-ser}, where the Python object is passed
to the \verb|json.dumps()| function, which also accepts an optional encoder
class as the \verb|cls| argument. Lines 2 and 3 of the listing show that the
\verb|None| value should not be serialized, because it is reserved for SQL
\verb|NULL| as previously explained.

\listing
{Python}
{The \code{value\_to\_string()} method of \code{JSONField} model field.}
{code:jsonfield-valuetostring}
{codes/4-jsonfield-valuetostring.py}

In addition to the serialization needed to store a Python object inside
\verb|JSONField| to the database, Django also has a serialization framework
that can be used for model objects. The framework can be used to dump model
objects into XML, JSON, and YAML files. The model objects are serialized by
calling the \verb|value_to_string()| method of their fields. Normally, the
method returns a string that is ready to be used by the serializer. However,
\verb|JSONField| represents JSON data in its Python built-in format that
can be serialized by the built-in serializers. Therefore, the method should
just return the Python object as-is and let the serializers handle the
serialization process. To achieve this, the \verb|value_to_string()| method
is overridden to just return the object using the \verb|value_from_object()|
method, as shown by line 2 of \autoref{code:jsonfield-valuetostring}.

\listing
{Python}
{The \code{from\_db\_value()} method of \code{JSONField} model field.}
{code:jsonfield-des}
{codes/4-jsonfield-des.py}

\verb|JSONField| is also designed to deserialize JSON-encoded strings to Python
objects in order to load the JSON data from the database. To convert database
values to Python objects, the \verb|from_db_value()| method can be overridden.
On some database systems that have a native JSON data type, the database driver
may already deserialize the value before Django receives it.\footnote{Only
PostgreSQL's Psycopg 2 driver does this at the time of writing, but this may
change in the future.} Lines 4 and 5 of \autoref{code:jsonfield-des} show that
the value is returned as-is, unless a custom decoder is used. In other cases,
the deserialization process is shown by line 7 of the listing, where
the database value is passed to the \verb|json.loads()| function, which also
accepts an optional decoder class as the \verb|cls| argument. Similar to
\verb|get_prep_value()|, the \verb|None| value is reserved for SQL \verb|NULL|,
as shown by lines 2 and 3 of the listing.

When extracting JSON string values, the database systems return them as
deserialized values (i.e. without JSON double quotes). Passing deserialized
strings to \verb|json.loads()| may cause \verb|json.JSONDecodeError| to be
raised. Therefore, as shown by lines 7 to 10 of \autoref{code:jsonfield-des},
the \verb|json.loads()| call is put inside a \verb|try ... except| block and
the value is directly returned if \verb|json.JSONDecodeError| is raised.

\listing
{Python}
{The \code{select\_format()} method of \code{JSONField}.}
{code:jsonfield-selectformat}
{codes/4-jsonfield-selectformat.py}

To cast JSON data to \verb|text| on the database level, the
\verb|select_format()| method is overridden in the \verb|JSONField| class. As
the name suggests, this method determines the format of the \verb|SELECT|
clause of the SQL query. The casting operation is needed in order to enable the
use of a custom decoder with a database system that automatically deserializes
JSON data into Python objects. Thus, the casting operation is performed only if
a custom decoder is used with a database backend that has the
\verb|has_native_json_field| feature flag set to \verb|True|, as shown by lines
2 to 6 of \autoref{code:jsonfield-selectformat}.

\listing
{Python}
{The \code{\_\_init\_\_()} method (constructor) of \code{JSONField}
model field.}
{code:jsonfield-init}
{codes/4-jsonfield-init.py}

The custom encoder and decoder are stored as instance attributes of the field,
as shown by lines 13 and 14 of \autoref{code:jsonfield-init}. It is important
to note that the encoder and decoder are subclasses of \verb|json.JSONEncoder|
and \verb|json.JSONDecoder|, rather than instances of those classes. In Python,
a class is a callable that returns an instance of the class, so the encoder and
decoder have to be callables. To enforce this requirement, the \verb|JSONField|
constructor checks whether the encoder and decoder are callables and raises
descriptive error messages if they are not, as shown by lines 5 to 12 of the
listing.

\listing
{Python}
{The \code{deconstruct()} method of \code{JSONField} model field.}
{code:jsonfield-dec}
{codes/4-jsonfield-dec.py}

In order to preserve the encoder and decoder classes in database migrations,
the \verb|deconstruct()| method needs to be overridden. The encoder and decoder
are defined as keyword arguments of the constructor. Therefore, the
\verb|deconstruct()| method is overridden to add the encoder and decoder
classes (if they are set, i.e. not \verb|None|) to the keyword arguments
dictionary, as shown by lines 3 to 6 of \autoref{code:jsonfield-dec}.

\listing
{Python}
{The \code{\_\_init\_\_()} method (constructor) of \code{JSONField} form field.}
{code:formfield-init}
{codes/4-formfield-init.py}

The \verb|JSONField| form field has also been updated to support custom encoder
and decoder. As with the model field, the encoder and decoder classes are
stored as instance attributes of the form field, as shown by lines 8 and 9 of
\autoref{code:formfield-init}. The encoder and decoder classes are used in
\verb|json.dumps()| and \verb|json.loads()| calls throughout the form field,
respectively.

\listing
{Python}
{The \code{prepare\_value()} and \code{has\_changed()} methods of
\code{JSONField} form field.}
{code:formfield-dumps}
{codes/4-formfield-dumps.py}

To utilize a custom encoder, some of the \verb|JSONField| form field methods
had to be updated. Line 4 of \autoref{code:formfield-dumps} shows the encoder
class used in the \verb|json.dumps()| call inside the \verb|prepare_value()|
method, which is used to prepare the value before it is shown to the user (e.g.
in an HTML form). Lines 12 and 13 show the encoder class used in the
\verb|json.dumps()| calls inside the \verb|has_changed()| method, which is used
to check whether the value inside the form field has changed.

\listing
{Python}
{The \code{to\_python()} and \code{bound\_data()} methods of \code{JSONField}
form field.}
{code:formfield-loads}
{codes/4-formfield-loads.py}

Some of the \verb|JSONField| form field methods also had to be updated to
support the use of a custom decoder. Line 4 of \autoref{code:formfield-loads}
shows the decoder class used in the \verb|json.loads()| call inside the
\verb|to_python()| method, which is used in the form field validation process.
If the value cannot be deserialized with the decoder, a \verb|ValidationError|
is raised, as shown by lines 5 to 10. Line 16 shows the decoder class used in
the \verb|json.loads()| call inside the \verb|bound_data()| method, which is
used to load the input data to the form field.

The form field only handles user input and does not interact with the database
directly. Thus, it is possible to use it with any database backend. As a
result, the form field has now been moved from the
\verb|django.contrib.postgres.forms| module to the \verb|django.forms| module.

\listing
{Python}
{The \code{formfield()} method of \code{JSONField} model field.}
{code:jsonfield-form}
{codes/4-jsonfield-form.py}

When model fields are included in a \verb|ModelForm|, Django automatically
generate form fields for them \cite{django:modelform}. The form fields are
generated by calling the \verb|formfield()| method of the model fields. In
order to use the encoder and decoder from the \verb|JSONField| model field in
the \verb|JSONField| form field, they need to be passed as keyword arguments,
as shown by lines 4 and 5 of \autoref{code:jsonfield-form}.

\listing
{Python}
{The \code{validate()} method of \code{JSONField} model field.}
{code:jsonfield-validate}
{codes/4-jsonfield-validate.py}

Model fields can define a built-in validation mechanism in the
\verb|validate()| method. The method is called by the field's \verb|clean()|
method as part of the model validation process of a \verb|ModelForm|. For
\verb|JSONField|, the \verb|validate()| method is overridden to provide a
basic validation mechanism by trying to serialize the field's value with the
given encoder, as shown by lines 3 and 4 of \autoref{code:jsonfield-validate}.
Lines 5 to 10 of the listing show that if the serialization fails, then a
\verb|ValidationError| is raised.

Now that the \verb|JSONField| class has been implemented, it is possible to
store and load JSON data to and from the database. As explained in the previous
chapter, there are also lookups and transforms that are specific for
\verb|JSONField|. The lookups and transforms extend the querying capabilities
for \verb|JSONField| by utilizing the functions and operators available on the
database systems. The implementation of the lookups and transforms will be
explained in the next section.

%-----------------------------------------------------------------------------%
\section{\code{JSONField} Lookups and Transforms}
%-----------------------------------------------------------------------------%

Lookups and transforms are part of Django's query expressions API. The API
consists of classes which instances can be compiled by Django's
\verb|SQLCompiler| objects. An \verb|SQLCompiler| object translates a query
expression by calling its \verb|as_vendorname()|\footnote{The \code{vendorname}
is the value of the \code{vendor} attribute in the \code{DatabaseWrapper} class
of the database backend, i.e. \code{postgresql}, \code{mysql}, \code{sqlite},
and \code{oracle}.} method (for vendor-specific implementation) if it exists,
and falls back to the \verb|as_sql()| method otherwise.

The lookups and transforms in this research are specific to \verb|JSONField|.
Normally, lookups and transforms are implemented in the
\verb|django.db.models.lookups| module. In the module, lookups are implemented
as subclasses of the \verb|Lookup| class, while transforms are implemented as
subclasses of the \verb|Transform| class. However, as the lookups and
transforms in this research are specific to \verb|JSONField|, they are
implemented as part of the \verb|django.db.models.fields.json| module.

The existing lookups for the PostgreSQL-only \verb|JSONField| serve as the
basis for the new implementation. The lookups are the containment lookups
(\verb|contains| and \verb|contained_by|), as well as the key existence
lookups (\verb|has_key|, \verb|has_keys|, and \verb|has_any_keys|). In addition
to that, the \verb|exact| lookup also has to be overridden in order to
support the JSON \verb|null| value.

The transforms are the key, index, and path transforms that extract the value
of JSON data at a given path. The index transform is just a special case of
the key transform where the key is an integer. Meanwhile, the path transform
can be viewed as a chain of key transforms. Therefore, the transforms are
unified as \verb|KeyTransform| in the existing implementation. The value
extracted by the transforms can be chained with some of the built-in lookups
and the \verb|JSONField|-specific lookups. As these transforms may return
values of different data types, some of the lookups also need to be modified
when they are used with on a \verb|KeyTransform|.

The lookups can be used directly on a \verb|JSONField| or on
\verb|KeyTransform|s applied to the field. When some of the lookups are used on
the transforms, they are implemented differently. Thus, before going in-depth
with the lookups implementation, it is better to go through the transforms
first.

\subsection{\code{JSONField} Transforms}

\listing
{Python}
{Parts of the \code{KeyTransform} class.}
{code:keytransform-1}
{codes/4-keytransform-1.py}

The key, index, and path
transforms are unified as the \verb|KeyTransform| class, which can be used to
extract the value of a \verb|JSONField| at a given path. The class includes the
{PostgreSQL} operators for extracting JSON values as shown by lines 2 and 3
of \autoref{code:keytransform-1}. Line 7 shows that the key name is normalized
as a string on initialization.

\listing
{Python}
{The \code{preprocess\_lhs()} method of the \code{KeyTransform} class.}
{code:keytransform-2}
{codes/4-keytransform-2.py}

To process a chain of \verb|KeyTransform|s that make up a path transform, the
\verb|preprocess_lhs()| method is created. Lines 4 to 8 of
\autoref{code:keytransform-2} show how the method separates the chain of
\verb|KeyTransform|s from the previous \verb|lhs|. The separation is done by
traversing the \verb|lhs| attribute of the \verb|KeyTransforms| until the
\verb|lhs| is not a \verb|KeyTransform|. While traversing, the key names that
construct the path are saved into a list. The method then compiles the original
\verb|lhs| and its parameters, and return them along with the list of
\verb|KeyTransforms|, as shown by lines 9 and 12. Lines 10 and 11 show that on
Oracle Database, any \verb|%| character in the key name is replaced by
\verb|%%| to escape string formatting (this will be explained later). In
addition, the method also has an \verb|lhs_only| parameter for optimization by
not saving and returning the key names if they are not needed, as shown by
lines 1, 2, 6, and 12.

\listing
{Python}
{The \code{as\_postgresql()} method of the \code{KeyTransform} class.}
{code:keytransform-3}
{codes/4-keytransform-3.py}

To compile a \verb|KeyTransform| into an SQL query expression, the
\verb|as_vendor()| method is used, e.g. \verb|as_postgresql()| for the
PostgreSQL backend. The method uses the \verb|preprocess_lhs()| method
described previously, as shown by line 2 of \autoref{code:keytransform-3}. If
there are multiple \verb|KeyTransform|s, the \verb|postgres_nested_operator|
(\verb|#>|) is used, as shown by lines 3 to 7 of the listing. Otherwise, the
\verb|postgres_operator| (\verb|->|) is used, as shown by lines 12 to 15. For
consistency, if there is only one \verb|KeyTransform| and the key is an
integer, it is assumed to be an array index by converting it to the \verb|int|
data type, as shown by line 9. The conversion is not needed for the nested
operator because for this operator, PostgreSQL automatically interprets integer
key names as array indices.

\listing
{Python}
{The \code{as\_sql()} method of the existing PostgreSQL-only \code{KeyTransform} class.}
{code:keytransform-0}
{codes/4-keytransform-0.py}

During this research, a security vulnerability was found in the existing
PostgreSQL-only implementation of \verb|KeyTransform|. Previously, the SQL
query expression was implemented in the \verb|as_sql()| method because it's
assumed that \verb|KeyTransform| is only used on PostgreSQL (as part of the
\verb|django.contrib.postgres| module). In the method, when there is only one
\verb|KeyTransform|, the key name is directly inserted into the string, rather
than passing it in the SQL parameters, as shown by lines 6, 8, and 9 of
\autoref{code:keytransform-0}. In addition, the key name is also not properly
quoted using \verb|json.dumps()|.

The approach in the existing implementation opened up a way for executing an
SQL injection by using a specially crafted string as the key name. The key name
is usually passed as a keyword argument in Python, which means that the string
can only contain alphanumeric characters and underscores. However, keyword
arguments can also be specified using a dictionary in Python, which allows any
string as a key name.

Using a malicious string as a key name in a dictionary that's passed as keyword
arguments, a user can execute an SQL injection. This vulnerability was reported
via email to the Django security team\footnote{Security issues in Django should
be reported to
\href{mailto:security@djangoproject.com}{security@djangoproject.com}, not the
public issue tracker.} and it was classified as CVE-2019-14234 \cite{cve} with
critical severity. To fix the vulnerability, a patch was issued for the Django
1.11, 2.1, and 2.2 releases \cite{django:securityrelease}. The implementation
of \verb|as_postgresql()| in this research has incorporated the patch, as shown
by lines 8 to 15 of \autoref{code:keytransform-3}.

\listing
{Python}
{The \code{compile\_json\_path()} function.}
{code:compilejsonpath}
{codes/4-compilejsonpath.py}

On other database systems, the JSON extraction functions use the JSONPath
notation to specify the path to be extracted. For reusability, the
\verb|compile_json_path()| function is created to convert a list of key names
into its JSONPath representation. The function works by iterating through the
list of key names and keeping a list of strings that make up the JSONPath. If
the key name is an integer, the string \verb|'[num]'| (where \verb|num| is the
integer) is appended to the list, as shown by lines 4, 5, 9, and 10 of
\autoref{code:compilejsonpath}. Otherwise, the key name is enclosed in double
quotes (so that it may contain spaces) by passing it through
\verb|json.dumps()|, as shown by lines 6 to 8. Additionally, the method also
has a \verb|include_root| parameter to control whether the resulting path
should include the root notation (\verb|$|), as shown by lines 1 and 2.
Finally, the list of strings is concatenated by joining the strings with an
empty string, as shown by line 11.

\listing
{Python}
{The \code{as\_mysql()} and \code{as\_sqlite()} methods of the \code{KeyTransform} class.}
{code:keytransform-4}
{codes/4-keytransform-4.py}

The \verb|as_vendor()| methods for MariaDB, MySQL, and SQLite are similar. The
methods utilize the \verb|preprocess_lhs()| method and
\verb|compile_json_path()| to construct the JSONPath notation of the key
transforms, as shown by lines 2, 3, 7, and 8 of \autoref{code:keytransform-4}.
To extract the value, the \verb|JSON_EXTRACT()| function is used, with
\verb|lhs| as the first argument and the JSONPath as the second argument, as
shown by lines 4 and 9.

\listing
{Python}
{The \code{as\_oracle()} method of the \code{KeyTransform} class.}
{code:keytransform-5}
{codes/4-keytransform-5.py}

Meanwhile, the \verb|as_oracle()| method needs some adjustments to work with
the behavior of Oracle Database. The method still uses the
\verb|preprocess_lhs()| method and \verb|compile_json_path()| function, as
shown by lines 2 and 3 of \autoref{code:keytransform-5}. However, to extract
the value, the \verb|JSON_QUERY()| and \verb|JSON_VALUE()| functions are
combined using the \verb|COALESCE()| function. This approach is needed as there
is no way to tell whether the value at the given path is a JSON object/array or
a scalar value. In addition, the path argument is added directly to the SQL
string, as shown by lines 5 and 6, because path expressions cannot be passed as
bind variables on Oracle Database \cite{oracle:jsonpath}.

\listing
{Python}
{Parts of the \code{Cast} class.}
{code:cast}
{codes/4-cast.py}

The JSON extraction functions are also useful for casting other fields to
\verb|JSONField| if the database system does not have a true data type for
JSON. To cast from one model field to another, Django provides the \verb|Cast|
class that can be used to perform casting on the database level. To support
casting to \verb|JSONField| on databases without a true data type for JSON, the
\verb|as_vendor()| methods of the \verb|Cast| class is overridden. The
\verb|JSON_EXTRACT()| function is used on MariaDB to force data to be
recognized as JSON, as shown by lines 6 to 8 of \autoref{code:cast}. Lines 14
to 16 show that on Oracle Database, the \verb|JSON_QUERY()| function is used.
For SQLite, using the \verb|JSON_EXTRACT()| function is not necessary as the
database can handle JSON data in its string form correctly.

\listing
{Python}
{The \code{KeyTextTransform} class.}
{code:keytexttransform}
{codes/4-keytexttransform.py}

Aside from \verb|KeyTransform|, the \verb|KeyTextTransform| class is also
implemented. The \verb|KeyTransform| class extracts JSON values with the
\verb|jsonb| data type on PostgreSQL. Meanwhile, some built-in lookups expect
the \verb|lhs| to be \verb|text|. Rather than casting the value to \verb|text|,
the \verb|KeyTextTransform| class works by using the \verb|->>| and \verb|#>>|
operators instead of the \verb|->| and \verb|#>| operators, as shown by lines 2
and 3 of \autoref{code:keytexttransform}. The \verb|as_vendor()| methods do not
need to be overridden for other database backends, because they already extract
JSON string values as \verb|text|.

\listing
{Python}
{The \code{KeyTransformFactory} class.}
{code:keytransformfactory}
{codes/4-keytransformfactory.py}

In order to use the transforms, a factory class (\verb|KeyTransformFactory|)
needs to be implemented. The key name of a \verb|KeyTransform| can be any
string as long as it doesn't clash with a registered lookup or transform.
Normally, transforms already have a predefined name for them, so their
\verb|__init__()| method does not have a parameter for defining the name. As
the key name of a \verb|KeyTransform| is defined dynamically, a factory class
is needed in order to encapsulate the key name information, as shown by line 4
of \autoref{code:keytransformfactory}. The key name will be passed as a
predefined argument when instantiating a \verb|KeyTransform| by calling the
factory object, as shown by lines 6 and 7 of the listing.

\listing
{Python}
{The \code{get\_transform()} method of the \code{JSONField} model field.}
{code:gettransform}
{codes/4-gettransform.py}

The factory instance is then registered to \verb|JSONField| by overriding the
\verb|get_transform()| method of \verb|JSONField| shown by
\autoref{code:gettransform}. The method works by trying to look recursively on
all parent classes for a registered transform with the given name and return it
if found, as shown by lines 2 to 4 of \autoref{code:gettransform}. If the
transform is not found, the method returns a \verb|KeyTransformFactory|
instance with the transform name as the key name, as shown by line 5 of the
listing.

As the transforms have been implemented, the lookups can now be implemented as
well. The lookups consist of the \verb|JSONField|-specific lookups and the
built-in lookups. The \verb|JSONField|-specific lookups are containment lookups
(\verb|contains| and \verb|contained_by|) and key existence lookups
(\verb|has_key|, \verb|has_keys|, and \verb|has_any_keys|). Meanwhile, the
built-in lookups are \verb|exact|, \verb|iexact|, \verb|isnull|,
\verb|icontains|, \verb|startswith|, \verb|istartswith|, \verb|endswith|,
\verb|iendswith|, \verb|regex|, \verb|iregex|, \verb|lt|, \verb|lte|,
\verb|gt|, and \verb|gte|.

\subsection{\code{JSONField} Lookups}

\listing
{Python}
{The \code{PostgresOperatorLookup} class.}
{code:pgop-lookup}
{codes/4-pgop-lookup.py}

In the existing PostgreSQL-only implementation, \verb|JSONField| lookups mostly
used a simple \verb|<lhs> <operator> <rhs>| format in the SQL queries sent to
the database (e.g. \verb|json_column @> '{"foo": "bar"}')|. To simplify the
implementation, a \code{PostgresSimpleLookup} class was created, so that the
lookups only needed to extend from the class and specify the operator. To
facilitate this research, the Django developers have moved this class from
\verb|django.contrib.postgres.lookups| to \verb|django.db.models.lookups| and
renamed the class to \verb|PostgresOperatorLookup|
\cite{gh-django:pgop-lookup}. By moving the class, the new implementation can
reuse the same logic without having to rely on the
\verb|django.contrib.postgres| module.

\listing
{Python}
{The new flags for the \code{JSONField} lookups in the \code{BaseDatabaseFeatures} class.}
{code:basedbfeatures2}
{codes/4-basedbfeatures2.py}

In addition to the database feature flags added to implement \verb|JSONField|,
there are some more of them added to implement the lookups, as shown by
\autoref{code:basedbfeatures2}. The first flag is \verb|has_json_operators|
that indicates whether the backend uses PostgreSQL-style JSON operators (e.g.
\verb|->|), which is set to \verb|False| and only overridden to \verb|True| for
PostgreSQL. The second flag is \verb|supports_json_field_contains| that
indicates whether the backend supports the containment lookups for
\verb|JSONField|. This flag is set to \verb|True| and overridden to
\verb|False| for SQLite and Oracle Database because they don't support the
containment lookups, as explained in the previous chapter. The last flag is the
\verb|json_key_contains_list_matching_requires_list| that indicates whether the
\verb|contains| lookup requires matching object structure. This flag is set
to \verb|False| and only overridden to \verb|True| for PostgreSQL as it is the
only system with the described behavior \cite{postgres:json}. Of the three
flags added above, the only flag used in the lookups implementation is
\verb|supports_json_field_contains|, while the others are only used in tests.

\listing
{Python}
{The \code{DataContains} class.}
{code:lookups-contains}
{codes/4-lookups-contains.py}

With the additional database feature flags added, the containment lookups can
now be implemented. One of the containment lookups is the \verb|contains|
lookup that matches JSON objects (\verb|lhs|) that contain the lookup value
(\verb|rhs|) as a subset. The lookup is implemented in the \verb|DataContains|
class shown by \autoref{code:lookups-contains}. On PostgreSQL, the lookup uses
the \verb|@>| operator, as shown by line 3 of the listing. On MariaDB and
MySQL, it uses the \verb|JSON_CONTAINS(lhs, rhs)| function, as shown by line
13. The lookup is not supported on other database backends, so the
\verb|NotSupportedError| is raised, as shown by lines 6 to 9.

\listing
{Python}
{The \code{ContainedBy} class.}
{code:lookups-contained_by}
{codes/4-lookups-contained_by.py}

The other containment lookup is the \verb|contained_by| lookup, implemented in
the \verb|ContainedBy| class shown by \autoref{code:lookups-contained_by}. The
lookup is the inverse of the \verb|contains| lookup, so it matches JSON objects
(\verb|lhs|) that are a subset of the lookup value (\verb|rhs|). The
implementation is similar to \verb|DataContains|, but the operator is changed
for PostgreSQL and the arguments switched for MariaDB and MySQL. On PostgreSQL,
the \verb|<@| is used. On MariaDB and MySQL, the \verb|JSON_CONTAINS(rhs, lhs)|
function is used, with the parameters switched, as shown by lines 13 and 14 of
the listing. As with \verb|DataContains|, the lookup is not supported by the
other database backends. With both of the lookups implemented, the containment
lookups are now complete.

The next set of lookups are the key existence lookups, which consist of the
\verb|has_key|, \verb|has_keys|, and \verb|has_any_keys| lookups. These
lookups have many similarities in their implementation. Thus, an abstraction
for these lookups are created as the \verb|HasKeyLookup| class.

\listing
{Python}
{The \code{as\_postgresql()} method of the \code{HasKeyLookup} class.}
{code:lookups-haskeylookup-1}
{codes/4-lookups-haskeylookup-1.py}

For PostgreSQL, the lookups make use of the \verb|?|, \verb|?&|, and \verb=?|=
operators, so the class extends from the \verb|PostgresOperatorLookup| class.
In the implementation of \verb|as_postgresql()|, the code also checks whether
the \verb|rhs| is a \verb|KeyTransform| rather than just a string. If so, then
the key names of the \verb|rhs| are retrieved and moved to the \verb|lhs|,
except the last one, as shown by lines 4 to 9 of
\autoref{code:lookups-haskeylookup-1}. After that, the \verb|rhs| is replaced
by the last key name, as shown by line 10. The adjustment is needed because
the key existence operators can only be used at the top-level of the data. To
check for keys at a certain depth, the \verb|lhs| needs to be extracted to
reach that depth in advance, so that the operators work on the extracted value.
Once the \verb|rhs| has been reduced to a single key, the method returns the
SQL expression produced by the \verb|PostgresOperatorLookup| class.

\listing
{Python}
{The \code{as\_sql()} method of the \code{HasKeyLookup} class.}
{code:lookups-haskeylookup-2}
{codes/4-lookups-haskeylookup-2.py}

For other database backends, they work similarly by using the JSONPath
notation. The JSONPath is used as the argument for the functions that check for
the existence of the JSONPath. To group the similarities found in the
\verb|as_vendor()| methods, the \verb|as_sql()| method is overridden, as shown
by \autoref{code:lookups-haskeylookup-2}.

The lookups can be used on a \verb|JSONField| or a \verb|KeyTransform|, so the
\verb|lhs| can be either of the two. If the \verb|lhs| is a
\verb|KeyTransform|, then all of the previous transforms are compiled into a
JSONPath, as shown by lines 6 to 10 of \autoref{code:lookups-haskeylookup-1}.
Otherwise, the \verb|lhs| is processed normally and the JSONPath is the root
(\verb|$|), as shown by lines 11 to 13.

The \verb|has_key| lookup accepts a key name (string) as the \verb|rhs|, but
the \verb|has_keys| and \verb|has_any_keys| lookup accepts a list or tuple of
key names. To unify the implementation, the \verb|rhs| is wrapped into a list
if it's not a list or tuple, as shown by lines 18 and 19 of
\autoref{code:lookups-haskeylookup-1}. Then, each key name in the \verb|rhs| is
compiled into a JSONPath that is prepended with the JSONPath from the lhs, as
shown by lines 20 to 30. After that, the JSONPaths are joined with a logical
operator defined by the class, as shown by lines 2 and 32 to 35. Finally, line
36 shows that the string result is returned along with the parameters.

\listing
{Python}
{The \code{as\_vendor()} methods of the \code{HasKeyLookup} class.}
{code:lookups-haskeylookup-3}
{codes/4-lookups-haskeylookup-3.py}

The differences between the database backends for \verb|HasKeyLookup| is
defined in the \verb|as_vendor()| methods. After the unification in
\verb|as_sql()| the \verb|as_vendor()| methods only need to define the
functions that are used on each database backend. MariaDB and MySQL use the
\verb|JSON_CONTAINS_PATH()| function, as shown by lines 1 to 4 of
\autoref{code:lookups-haskeylookup-3}. SQLite uses the \verb|JSON_TYPE()|
function with the \verb|IS NOT NULL| condition, as shown by lines 6 to 9.
Oracle Database uses the \verb|JSON_EXISTS()| function, as shown by lines 11 to
14. Additionally, the JSONPath needs to be added directly into the SQL
expression string, as shown by line 15, because it cannot be passed as bind
variables.

\listing
{Python}
{The \code{HasKey} class.}
{code:lookups-haskey}
{codes/4-lookups-haskey.py}

\listing
{Python}
{The \code{HasKeys} class.}
{code:lookups-haskeys}
{codes/4-lookups-haskeys.py}

\listing
{Python}
{The \code{HasAnyKeys} class.}
{code:lookups-hasanykeys}
{codes/4-lookups-hasanykeys.py}

After the \verb|HasKeyLookup| class is defined, the concrete classes for the
lookups only need to define the operators. The \verb|has_key| lookup is defined
in the \verb|HasKey| class and uses the \verb|?| PostgreSQL operator with no
logical operator for the other database backends, as shown by
\autoref{code:lookups-haskey}. The \verb|has_keys| lookup is defined in the
\verb|HasKeys| class and uses the \verb|?&| PostgreSQL operator and the
\verb|AND| logical operator for the other database backends, as shown by
\autoref{code:lookups-haskeys}. The \verb|has_any_keys| lookup is defined in
the \verb|HasAnyKeys| class and uses the \verb=?|= PostgreSQL operator and the
\verb|OR| logical operator for the other database backends, as shown by
\autoref{code:lookups-hasanykeys}.

In addition, the lookups need to prevent the \verb|rhs| from being serialized
into a JSON string because it would incorrectly add double quotes to the key
name. To do so, the \verb|prepare_rhs| flag is set to \verb|False| for the
\verb|HasKey| class. For the \verb|HasKeys| class, the \verb|rhs| is only
normalized (rather than serialized) as a list of strings by overriding the
\verb|get_prep_lookup()| method, as shown by lines 6 and 7 of
\autoref{code:lookups-haskeys}. The \verb|HasAnyKeys| class extends from the
\verb|HasKeys| lookup, so the overridden \verb|get_prep_lookup()| method is
inherited. As the concrete classes have been implemented, the key existence
lookups are now complete.

\listing
{Python}
{The \code{JSONExact} class.}
{code:lookups-jsonexact}
{codes/4-lookups-jsonexact.py}

Aside from the \verb|JSONField|-specific lookups, the \verb|exact| lookup also
needs to be modified in order to work with \verb|JSONField|. The lookup is
implemented in the \verb|JSONExact| class that extends \verb|lookups.Exact|, as
shown by line 1 of \autoref{code:lookups-jsonexact}. Line 2 shows that the
class has the \verb|can_use_none_as_rhs| flag set to \verb|True|. This override
prevents Django from automatically swapping the \verb|exact=None| lookup with
the \verb|isnull=True| lookup, which would turn the lookup into a query for SQL
\verb|NULL|. By disabling the automatic swapping, it is possible to query for
JSON \verb|null| value using \verb|exact=None|.

The \verb|process_lhs()| and \verb|process_rhs()| methods of the \verb|exact|
lookup are also overridden. The \verb|process_lhs()| is modified so that the
\verb|lhs| is wrapped with the \verb|JSON_TYPE()| function on SQLite if the
\verb|rhs| is \verb|None|, as shown by lines 6 to 9 of
\autoref{code:lookups-jsonexact}. On the other hand, the \verb|process_rhs()|
is modified by replacing \verb|None| with \verb|'null'| for all database
backends, as shown by lines 14 and 15. To make MariaDB and MySQL treat the
\verb|rhs| as a JSON value, the \verb|JSON_EXTRACT()| function is used, as
shown by lines 16 to 18.

\listing
{Python}
{The \code{field\_cast\_sql()} and \code{lookup\_cast()} methods of the
Oracle Database backend's \code{DatabaseOperations} class.}
{code:oracle-operations}
{codes/4-oracle-operations.py}

Once again, the \verb|DatabaseOperations| of the Oracle Database backend also
needs to be modified. As JSON data is stored using the \verb|NCLOB| data type,
directly including the value in the \verb|WHERE| clause of an SQL query is not
supported \cite{oracle:comparison}. To work around this limitation, the LOB
object is read using the \verb|DBMS_LOB.SUBSTR()| function, as shown by lines 8
and 9 of \autoref{code:oracle-operations}. To avoid unnecessary calls to
\verb|DBMS_LOB.SUBSTR()| in other situations, an additional check is added to
the \verb|field_cast_sql()| method, as shown by line 2. After applying these
modifications to the backend and the lookup class, the \verb|exact| lookup is
now compatible with \verb|JSONField|.

\listing
{Python}
{The registration of \code{JSONField} lookups.}
{code:register-jsonfield}
{codes/4-register-jsonfield.py}

Before custom lookups can be used on a model field, they need to be registered
using the \verb|register_lookup()| method of the field. The customized lookups
for \verb|JSONField| are the containment lookups, the key existence lookups,
and the \verb|exact| lookup. Therefore, the \verb|DataContains|,
\verb|ContainedBy|, \verb|HasKey|, \verb|HasKeys|, \verb|HasAnyKeys|, and
\verb|JSONExact| classes are registered on \verb|JSONField|, as shown by
\autoref{code:register-jsonfield}. After registering the lookups, they can now
be used on \verb|JSONField|. However, these lookups and other built-in lookups
still need to be modified in order to be used on \verb|KeyTransform|s, as
explained in the next subsection.

\subsection{\code{KeyTransform} Lookups}

The \verb|KeyTransform| class can be chained with \verb|JSONField|-specific
lookups and other built-in lookups. The \verb|KeyTransform| class works by
extracting the value of some JSON data at a given path. The extracted value
may be of any type: boolean, number, string, object, array, or \verb|null|.
In addition, the path may also be nonexistent. To implement the lookups on
\verb|KeyTransform|s, the different possibilities of the extracted value must
be taken into account.

\listing
{Python}
{The \code{process\_lhs()} method of \code{KeyTransformExact}.}
{code:lookups-keytransformexact-1}
{codes/4-lookups-keytransformexact-1.py}

One of the lookups that needs to be modified is the \verb|exact| lookup. The
lookup is implemented in the \verb|KeyTransformExact| class that extends
\verb|JSONExact|. As with \verb|JSONExact|, the \verb|process_lhs()| method is
overridden to wrap the \verb|lhs| with the \verb|JSON_TYPE()| function when
querying JSON \verb|null| value, as shown by line 10 of
\autoref{code:lookups-keytransformexact-1}. However, as the \verb|rhs_params|
are retrieved from the \verb|process_rhs()| of \verb|JSONExact|, the value has
changed from \verb|[None]| to \verb|['null']|, as shown by line 6. In addition,
the \verb|preprocess_lhs()| method is used to retrieve the original \verb|lhs|
before the \verb|KeyTransform|s are applied, as shown by lines 7 to 9 of the
listing.

\listing
{Python}
{The \code{process\_rhs()} method of \code{KeyTransformExact}.}
{code:lookups-keytransformexact-2}
{codes/4-lookups-keytransformexact-2.py}

The \verb|process_rhs()| method of \verb|KeyTransformExact| also needs to be
overridden. The \verb|rhs| of \verb|KeyTransformExact| can also be a
\verb|KeyTransform| expression, in which case the \verb|process_rhs()| of the
built-in \verb|exact| lookup is used, as shown by
\autoref{code:lookups-keytransformexact-2}. In other cases, the \verb|rhs|
needs to be wrapped in a JSON extraction function to be treated as JSON. On
Oracle Database, each value of the \verb|rhs_params| is wrapped inside a JSON
object at the key \verb|value|, as shown by lines 13 to 16. If the value is a
JSON object or array, the \verb|JSON_QUERY()| function is used. Otherwise, the
\verb|JSON_VALUE()| function is used, as shown by lines 9 to 12. Meanwhile on
SQLite, the \verb|JSON_EXTRACT()| function is used (unless the value is
\verb|null|), as shown by lines 19 to 24.

\listing
{Python}
{The \code{as\_oracle()} method of \code{KeyTransformExact}.}
{code:lookups-keytransformexact-3}
{codes/4-lookups-keytransformexact-3.py}

In addition, the \verb|as_oracle()| method of \verb|KeyTransformExact| is also
overridden to work with the JSON \verb|null| value. On Oracle Database,
querying for the JSON \verb|null| value works by checking that the key exists
and that the value (when extracted) is \verb|NULL|. The checks are implemented
by combining the \verb|has_key| lookup and the \verb|isnull| lookup, as shown
by \autoref{code:lookups-keytransformexact-3}.

\listing
{Python}
{The \code{KeyTransformIsNull} class.}
{code:lookups-keytransformisnull}
{codes/4-lookups-keytransformisnull.py}

The next lookup to be modified is the \verb|isnull| lookup, which is
implemented in the \verb|KeyTransformIsNull| class. When used on a
\verb|KeyTransform|, the \verb|isnull=True| lookup matches objects that do not
have the path specified by the \verb|KeyTransform|s, and vice versa. On
PostgreSQL, MariaDB, and MySQL, the JSON extraction functions used in
\verb|KeyTransform| already return the correct result for the \verb|isnull|
lookup. However, on SQLite and Oracle Database, the JSON extraction functions
return SQL \verb|NULL| when extracting JSON \verb|null|. The behavior affects
the \verb|isnull=False| lookup as it doesn't match objects where the path
exists but the value is \verb|null|. This issue can be fixed by replacing the
\verb|isnull=False| lookup with a \verb|HasKey| lookup performed on the
\verb|lhs|, as shown by lines 3, 4, 8, and 9 of
\autoref{code:lookups-keytransformisnull}.

The remaining lookups for \verb|KeyTransform| can be divided into two groups:
the text lookups and the numeric lookups. The text lookups are the
\verb|iexact|, \verb|icontains|, \verb|startswith|, \verb|istartswith|,
\verb|endswith|, \verb|iendswith|, \verb|regex|, and \verb|iregex| lookups. The
text lookups that start with \verb|i| are case-insensitive. Meanwhile, the
numeric lookups are the \verb|lt|, \verb|lte|, \verb|gt|, and \verb|gte|
lookups.

To use the built-in text lookups, the \verb|lhs| needs to be in the form of an
SQL string. On PostgreSQL, JSON values can be extracted as \verb|text| (rather
than \verb|jsonb|) using the \verb|->>| and \verb|#>>| operators, which have
been implemented in the \verb|KeyTransformText| class. On MariaDB and MySQL,
after using the \verb|JSON_EXTRACT()| function to extract the value, JSON
strings need to be unquoted using the \verb|JSON_UNQUOTE| function. On SQLite,
the \verb|JSON_EXTRACT()| function automatically unquotes JSON strings, so no
modification is needed. On Oracle Database, the \verb|JSON_VALUE()| function
also automatically unquotes JSON strings, so there is no modification needed as
well.

\listing
{Python}
{The \code{KeyTransformTextLookupMixin} class.}
{code:mixins-keytransformtextlookup}
{codes/4-mixins-keytransformtextlookup.py}

In order to use \verb|KeyTransformText| for lookups that expect the \verb|lhs|
as \verb|text|, the \verb|KeyTransformTextLookupMixin| class is created. The
mixin works by replacing the \verb|KeyTransform| with a \verb|KeyTextTransform|
in the \verb|__init__()| method, as shown by lines 8 to 12 of
\autoref{code:mixins-keytransformtextlookup}. If the class is instantiated with
a transform other than a \verb|KeyTransform|, a \verb|TypeError| is raised, as
shown by lines 3 to 7.

\listing
{Python}
{The \code{lookup\_cast()} method in the MySQL backend's
\code{DatabaseOperations} class.}
{code:mysql-operations}
{codes/4-mysql-operations.py}

To utilize the \verb|JSON_UNQUOTE()| function when performing lookups that
expect a string \verb|lhs| on MariaDB and MySQL, the \verb|DatabaseOperations|
class of the MySQL backend is modified by overriding the \verb|lookup_cast()|
method. If the \verb|lookup_type| argument of the method is one of the text
lookups and it is performed on a \verb|JSONField|, the method returns the
\verb|JSON_UNQUOTE()| function to be interpolated, as shown by lines 3 to 8 of
\autoref{code:mysql-operations}. In addition, as MariaDB does not have a native
JSON data type, it may require the \verb|JSON_UNQUOTE()| function regardless of
the lookup. Thus, the condition for MariaDB is added accordingly.

\listing
{Python}
{The \code{CaseInsensitiveMixin} class.}
{code:mixins-caseinsensitivemixin}
{codes/4-mixins-caseinsensitivemixin.py}

The text lookups also include case-insensitive lookups, but comparison of JSON
strings on MariaDB and MySQL is case-sensitive by default.\footnote{MariaDB and
MySQL handle strings in JSON context using the \code{utf8mb4\_bin} collation,
which is case sensitive.} To allow case-insensitive comparison of JSON values
on MariaDB and MySQL, the \verb|CaseInsensitiveMixin| class is created. The
mixin works by converting the \verb|lhs| and \verb|rhs| to lowercase using the
\verb|LOWER()| function, as shown by
\autoref{code:mixins-caseinsensitivemixin}.

\listing
{Python}
{The text lookup classes for \code{KeyTransform}.}
{code:lookups-keytransformtext}
{codes/4-lookups-keytransformtext.py}

With the necessary mixins created and the backend modification applied, the
text lookups for \verb|KeyTransform| can be implemented. The lookups are
implemented by extending the mixins and the built-in lookups accordingly, as
shown by \autoref{code:lookups-keytransformtext}. All of the lookup classes
extend the \verb|KeyTransformTextLookupMixin| class and their respective
built-in lookups. For case-insensitive lookups, the classes also extend the
\verb|CaseInsensitiveMixin| class. There are no methods or attributes to be
overridden, as Python automatically resolve any nonexistent properties to the
superclasses using its multiple inheritance logic.

\listing
{Python}
{The \code{KeyTransformNumericLookupMixin} class.}
{code:mixins-keytransformnumeric}
{codes/4-mixins-keytransformnumeric.py}

The remaining lookups for \verb|KeyTransform| are the numeric lookups, which
may or may not require the \verb|rhs| to be in a numeric type. To implement
the lookups, the \verb|KeyTransformNumericLookupMixin| class is created. For
database systems that have a native JSON data type, the \verb|rhs| should be
passed in its serialized form, so there is no modification needed. For other
database systems, the \verb|rhs| should be in a numeric (deserialized) type.
The mixin works by conditionally deserializing the rhs using the
\verb|json.loads()| function, as shown by
\autoref{code:mixins-keytransformnumeric}.

\listing
{Python}
{The numeric lookup classes for \code{KeyTransform}.}
{code:lookups-keytransformnumeric}
{codes/4-lookups-keytransformnumeric.py}

As with the text lookups, the numeric lookups can be implemented by extending
the mixins and built-in lookups. All of the numeric lookups extend from the
\verb|KeyTransformNumericLookupMixin| to ensure the \verb|rhs| is correct.
After that, the classes can just extend from the built-in lookups without
having to override any properties, as shown by
\autoref{code:lookups-keytransformnumeric}.

\listing
{Python}
{The registration of \code{KeyTransform} lookups.}
{code:register-keytransform}
{codes/4-register-keytransform.py}

With all of the necessary classes implemented, the lookups are now compatible
with \verb|KeyTransform|s. As with the \verb|JSONField| lookups, they need to
be registered on \verb|KeyTransform|. \autoref{code:register-keytransform}
shows that all of the lookups are registered to \verb|KeyTransform| using the
\verb|register_lookup()| method. The \verb|JSONField|-specific lookups can be
used without explicitly registering them to \verb|KeyTransform|, because
the output field of \verb|KeyTransform| is also \verb|JSONField|.

\clearchapter
%-----------------------------------------------------------------------------%
\chapter{\babLima}
%-----------------------------------------------------------------------------%

In the previous chapter, the implementation of \verb|JSONField| has been
explained, which includes basic validation of the JSON data. However, the basic
validation only checks the syntax and not the semantics of the data. To
validate the semantics, additional checking can be implemented using Django's
validation features on the application-level, which is demonstrated in this
chapter.

%-----------------------------------------------------------------------------%
\section{Scenario}
%-----------------------------------------------------------------------------%

Before going through how the validation works, a small Django project with an
e-commerce scenario is created for demonstration. The project includes two
models: \verb|Product| and \verb|Cart|, both of which utilize \verb|JSONField|
in their definitions. In the project, the website owner can create multiple
\verb|Product| objects as part of the website's catalogue. Meanwhile, visitors
can add multiple \verb|Product| objects into their \verb|Cart|. For simplicity,
the \verb|Cart| object only holds some essential information of the
\verb|Product| objects to be shown in the shopping cart summary.

Rather than using a relational field such as \verb|ManyToManyField| to store
the \verb|Product| objects in the \verb|Cart|, a \verb|JSONField| is used. The
\verb|JSONField| stores a \verb|list| of simplified \verb|Product| data. In
Python and JSON, \verb|list|s (arrays in JSON) can contain elements with
different data types, which means that any arbitrary data can be inserted into
a \verb|list|. Without having any validation rules, the list in the
\verb|JSONField| may hold invalid \verb|Product| data. Thus, validation rules
will be created to ensure that the list only contains \verb|Product| data with
valid identifiers.

\listing
{Python}
{The \code{Product} model.}
{code:product}
{codes/5-product.py}

The \verb|Product| model shown by \autoref{code:product} represents a product
on an e-commerce website. The \verb|Product| model has an \verb|id| (unique for
each product), a \verb|name|, a \verb|price|, a \verb|stock_qty| (quantity of
the product in stock), and additional \verb|data|. The \verb|data| is stored as
a \verb|JSONField| so that each product can define its own attributes in the
form of a JSON object (\verb|dict| in Python).

\listing
{Python}
{The \code{Cart} model.}
{code:cart}
{codes/5-cart.py}

Meanwhile, the \verb|Cart| model shown by \autoref{code:cart} represents a
shopping cart that is linked to a user and it can hold a list of products. The
list of products is stored in a \verb|JSONField| in the form of a JSON array
(\verb|list| in Python). In addition, the \verb|JSONField| is supplied with two
validators, \verb|validate_cart_products_list| and
\verb|validate_cart_product_ids_exist|. The first validator validator is used
to validate that the \verb|products| field is a \verb|list|. Meanwhile, the
second validator ensures that the \verb|id|s in the list of \verb|products|
also exist in the database. For simplicity, the validator only checks the
\verb|id| and not the other \verb|Product| attributes, as they can be retrieved
as long as the \verb|Product| with the same \verb|id| exists in the database.
These validators will be explained in more detail in the next section.

\listing
{Python}
{The \code{add\_product()} method of the \code{Cart} model.}
{code:addproduct}
{codes/5-addproduct.py}

The \verb|Cart| model also has an \verb|add_product()| method to ease the
process of adding a \verb|Product| object to the list of \verb|products|. For
simplicity, the list only holds the \verb|id|, \verb|name|, and \verb|price| of
the \verb|Product| objects, as well as the quantity of each \verb|Product|
object that is added to the cart. The \verb|id| is stored as a hex string
(rather than a \verb|UUID| object) and the \verb|price| is stored as a
\verb|float| (rather than a \verb|Decimal| object) so that the \verb|JSONField|
does not need a custom encoder. Whenever a \verb|Product| is added, the method
iterates through the list of \verb|products| to find the \verb|Product| that
has the same \verb|id| and increments the quantity if it is found, as shown by
lines 15 to 18 of \autoref{code:addproduct}. Otherwise, the data is just
appended to the list, as shown by lines 19 to 26. The list of \verb|products|
will be validated using the validators, as explained in the next section.

%-----------------------------------------------------------------------------%
\section{Validator functions}
%-----------------------------------------------------------------------------%

When defining the fields of a model or a form, Django allows the use of
validators that can be used to validate the data within the field. A validator
is a callable object that takes a value and raises a \verb|ValidationError| if
the value is not valid according to the logic defined by the validator
\cite{django:validators}. To apply validators to a model field or a form field,
they are passed to the field's constructor using the \verb|validators|
argument.

In addition to the built-in validators (e.g. \verb|RegexValidator|,
\verb|EmailValidator|), Django also allows programmers to create their own
validators. A validator is a callable, which means that it can be a function or
an instance of a class that defines the \verb|__call__()| method. In the
previous section, two custom validators are used to validate the list of
\verb|products| (a \verb|JSONField|): \verb|validate_cart_products_list| and
\verb|validate_cart_product_ids_exist|.

\listing
{Python}
{The \code{validate\_cart\_products\_list()} function.}
{code:validatelist}
{codes/5-validatelist.py}

The first validator function is the \verb|validate_cart_products_list()|
function shown by \autoref{code:validatelist}. The function works by checking
whether the \verb|products| value is an instance of \verb|list| and raising a
\verb|ValidationError| if it is not a \verb|list|. This validator function can
be seen as a basic schema validator for the JSON data that ensures the data is
a JSON array. To validate against a more complex schema, programmers can use a
third-party library such as
jsonschema\footnote{\url{https://github.com/Julian/jsonschema}} or
Cerberus\footnote{\url{https://github.com/pyeve/cerberus}}. If the library
indicates that the data is invalid, then a \verb|ValidationError| shall be
raised.

\listing
{Python}
{The \code{validate\_cart\_product\_ids\_exist()} function.}
{code:validateids}
{codes/5-validateids.py}

The second validator function is \verb|validate_cart_product_ids_exist()| that
ensures the \verb|id|s of the \verb|Product| data inside the list of
\verb|products| also exist in the database. To prevent errors while validating
the list of \verb|products|, the validator immediately returns if the
\verb|products| value is not a list, as shown by lines 2 and 3 of
\autoref{code:validateids}. Otherwise, the validator takes the \verb|id|s of
each product in the cart into a list of \verb|product_ids| (line 6).

After the \verb|product_ids| list is created, a query is executed to look for
the \verb|Product| objects where the \verb|id| is in the list of
\verb|product_ids| (line 8). To retrieve the \verb|id|s back as strings (not as
\verb|UUID| objects), the \verb|Cast| function is used to cast the \verb|id|
into a \verb|CharField| (lines 9 and 10). The \verb|values_list()| method is
used so that the query only retrieves the \verb|id|s without the other
information of the \verb|Product| model (line 11). Line 13 of the listing shows
that the set of \verb|product_ids| is subtracted by the set of \verb|id|s
retrieved from the database (\verb|valid_ids|). If the subtraction results in a
non-empty set, that means the list of \verb|products| contain \verb|Product|
information with invalid \verb|id|s, so a \verb|ValidationError| is raised, as
shown by lines 14 to 17 of the listing. As all of the validators have been
implemented, they can now be used to validate the list of \verb|products|.

%-----------------------------------------------------------------------------%
\section{Demonstration}
%-----------------------------------------------------------------------------%

To use the validators, they need to be run manually before saving the model
object, because the method does not run them automatically. To run the
validators, the \verb|clean_fields()| method of the model object should be
called. The method can be called either directly or indirectly by calling the
\verb|full_clean()| method of the model object, which calls
\verb|clean_fields()| as part of its process.

\begin{figure}
	\centering
    \includegraphics[width=0.90\textwidth]{pics/validation0.png}
	\caption{The initialization for the validation demonstration.}
	\label{fig:validation0}
\end{figure}

Before demonstrating the validators in action, some model objects are created,
as shown by \autoref{fig:validation0}. The first object is a \verb|User| object
that is assigned to the \verb|user| variable, which will be assigned to a
\verb|Cart| object. The second object is a \verb|Product| model object that is
assigned to the \verb|product| variable, which will be added to a \verb|Cart|
object. Both of these objects are created with the \verb|create()| method,
so they are immediately saved to the database. After these objects are created,
the validators are demonstrated below.

\begin{figure}
	\centering
    \includegraphics[width=0.90\textwidth]{pics/validation1.png}
	\caption{The \code{validate\_cart\_products\_list()} validator in action.}
	\label{fig:validation1}
\end{figure}

To run the validators before the \verb|Cart| model object is saved to the
database, an instance of the \verb|Cart| class is created (rather than using
the \verb|create()| method). The instance is linked to the \verb|user| created
previously and is initialized with the dictionary \verb|{'products': []}| as
the top-level value for the \verb|products| field. When calling the
\verb|clean_fields()| method, a \verb|ValidationError| is raised, as shown by
\autoref{fig:validation1}. The error message (that comes from the
\verb|validate_cart_products_list| validator) indicates that the
\verb|products| field value should be a \verb|list|, not a dictionary. After
changing the \verb|products| field's top-level value to an empty \verb|list|
(\verb|[]|), calling the \verb|clean_fields()| method again does not raise any
errors.

\begin{figure}
	\centering
    \includegraphics[width=0.90\textwidth]{pics/validation2.png}
	\caption{The \code{validate\_cart\_product\_ids\_exist} validator in action.}
	\label{fig:validation2}
\end{figure}

The next validator is the \verb|validate_cart_product_ids_exist()| function.
Before demonstrating this validator, the previously made \verb|product| object
is added to the \verb|cart|. If the \verb|clean_fields()| method is called, no
error is raised because the \verb|product| exists in the database, as shown by
input 11 of \autoref{fig:validation2}. After that, a new \verb|Product|
instance is created without saving it to the database (input 12). If the
instance is added to the \verb|cart| (input 14) and the \verb|clean_fields()|
method is called, a \verb|ValidationError| is raised (input 15). The error
message indicates that the invalid ID is equal to the new \verb|Product|
instance \verb|id| (input and output 13). The error is raised because the
new instance does not exist in the database yet.

\begin{figure}
	\centering
    \includegraphics[width=0.90\textwidth]{pics/validation3.png}
	\caption{The \code{Cart} object is saved after passing the validations.}
	\label{fig:validation3}
\end{figure}

If the new \verb|Product| instance is saved, calling the \verb|clean_fields()|
method again does not raise an error, as shown by inputs 16 and 17 of
\autoref{fig:validation3}. As the list of \verb|products| in the \verb|cart|
has been validated, it can now be safely saved to the database. To save the
\verb|Cart| object, the \verb|save()| method is called (input 18). To verify
that the object is successfully saved, the \verb|refresh_from_db()| method is
called (input 19). As shown by input and output 20, the list of \verb|products|
were successfully stored to and retrieved from the database.

In this chapter, JSON data validation for a \verb|JSONField| has been
explained. The validation is implemented by creating custom validator functions
using and supplying them to the field using the \verb|validators| argument to
the field's constructor. To run the validators, the \verb|clean_fields()|
method of the field is called. If the data is not valid, the validators raise
\verb|ValidationError|s, which can be caught to prevent invalid data from being
saved to the database. In the next chapter, the \verb|JSONField| implementation
will be evaluated on multiple database systems.

\clearchapter
%---------------------------------------------------------------
\chapter{\kesimpulan}
%---------------------------------------------------------------
This chapter discusses the conclusions of this research and suggestions for
future research.

%---------------------------------------------------------------
\section{Conclusions}
%---------------------------------------------------------------

The primary objective of this research is to implement a new \verb|JSONField|
that can be used on all database backends supported by Django. The
\verb|JSONField| implementation includes the model field, form field, lookups,
and transforms. The implementation should handle the differences in how JSON
data is managed by all of the database systems, which consist of PostgreSQL,
MariaDB, MySQL, SQLite, and Oracle Database.

To handle the differences between the database systems, some adjustments to the
database backends were needed. The adjustments include the additions of new
feature flags in the \verb|DatabaseFeature| classes of the database backends.
The flags indicate the different behaviors of each database system, which can
be used to determine how the implementation on the database system should be
created. Some adjustments were also made to the \verb|DatabaseWrapper| classes
to define the data types and SQL \verb|CHECK| constraints on the database
level. Some database backends also require some adjustments to the
\verb|DatabaseOperations| class to define some casting operations for JSON
data.

After creating the database backend adjustments, \verb|JSONField| and its
lookups and transforms could be implemented. The \verb|JSONField| model field
was implemented by serializing the data before it is sent to the database and
deserializing the data when it is retrieved from the database. On the other
hand, the \verb|JSONField| form field could be used without any modifications
as it does not interact with the database directly. Both the model field and
form fields were updated to support custom encoder and decoder for
serialization and deserialization. The lookups and transforms were implemented
by utilizing the JSON functions that are available on the database systems.

The secondary objective of this research is to provide examples of how
validation rules can be applied to a \verb|JSONField|. For demonstration, a
small Django project with an e-commerce scenario was created. The project
includes two models, \verb|Product| and \verb|Cart|, that represent a product
and a shopping cart on an e-commerce website. The \verb|Cart| model has a
\verb|JSONField| that uses validators to validate the data. The validators were
added using the Django's validation feature. By utilizing the validators, JSON
data in a model field could be validated by calling the \verb|clean_fields()|
method of the model instance and catching the \verb|ValidationError| exception.

To ensure that the new implementation of \verb|JSONField| is consistent on all
database backends, \verb|JSONField| tests were run on all database systems. The
tests included test cases for the new features and documented behaviors, such
as the custom decoder support and the \verb|JSONField| behavior when handling
JSON \verb|null| and SQL \verb|NULL| values. In addition, the \verb|JSONField|
tests also incorporated the tests from the previous PostgreSQL-only
implementation to ensure that backward compatibility is preserved. After
passing the tests, the previous implementation was deprecated by creating
replicas of the classes and raising system warnings about the deprecation.

%---------------------------------------------------------------
\section{Suggestions}
%---------------------------------------------------------------
Based on the results of this research, some suggestions for future research are
as follow:

\begin{enumerate}
	\item The Query Expression API in Django supports a variety of use cases
		  through the use of built-in and custom query expressions. While the
		  tests in this research have included the common use cases, it is not
		  feasible to test each and every possible use case with
		  \verb|JSONField|. Thus, more complex use cases of \verb|JSONField|
		  and Django's ORM tool on all database backends may be a subject of
		  future research.
	\item Some database systems support partial updates to JSON data through
		  the use of database functions for more efficient write operations on
		  JSON columns. The implementation in this research does not make use
		  of the partial update feature, which means that JSON data is always
		  updated by rewriting the whole data. Future research on
		  \verb|JSONField| can improve upon this research by utilizing the
		  partial update feature on database systems that support it.
	\item This research has provided some examples of JSON data validation,
		  including a basic example of how the schema of a \verb|JSONField| can
		  be validated. For more complex schemas, third-party packages can be
		  utilized to create the validation rules. Integrating the third-party
		  packages with Django's validators may be explored in future
		  research.
\end{enumerate}

\clearchapter

%
% Daftar Pustaka
\include{pustaka}
\clearchapter

%
% Lampiran
%
\begin{appendix}
	\include{markLampiran}
	\clearchapter
	\setcounter{page}{2}
	\include{lampiran}
\end{appendix}

\end{document}
