%-----------------------------------------------------------------------------%
\chapter{\babDua}
%-----------------------------------------------------------------------------%

This chapter focuses on building a conceptual basis by reviewing existing
literature and documentation of the underlying concepts of this research.

%-----------------------------------------------------------------------------%
\section{Web Framework}
%-----------------------------------------------------------------------------%

A web framework is a software framework that is designed to support the
development of web applications. It consists of reusable components that solve
common problems in web application development. Most web frameworks provide
safe defaults to prevent security problems in the web application. This
convenience helps programmers build web applications faster and safer. Some web
frameworks also help maintain better application structure by enforcing
software design patterns.

The common components of a web framework are the dispatcher, the decoder, the
generator, and the store \cite{schwarz_webframework}. The dispatcher maps
Uniform Resource Locators (URLs) to application code that is invoked when an
HTTP request is received. The decoder decodes the client request, which allows
the web application to read parameters, headers, and request body data sent as
part of the request. The generator constructs the output that is sent as a
response to the client. The store holds inter-request data, which is usually
stored in a database system.

%-----------------------------------------------------------------------------%
\section{Django and Object-Relational Mapping}
%-----------------------------------------------------------------------------%

Django is a free and open-source web framework written in Python \cite{django}.
Its overall philosophies are loose coupling, less code, quick development,
don't repeat yourself (DRY), explicit is better than implicit, and consistency
\cite{django:philosophies}. Django follows what it calls the
model-template-view architectural pattern, which shares similarities to the
model-view-controller pattern \cite{django:faq}. It consists of an
object-relational mapping (ORM) system that handles the interaction with
database systems, a template system that defines how the data looks like to the
user, and a view system that defines which data is presented to the user.

Object-relational mapping (ORM) is a programming technique that allows its
users to query and manipulate data stored in a relational database using an
object-oriented paradigm \cite{linskey:orm}. Object query, load, and delete
operations are translated into SQL \code{SELECT} and \code{DELETE} statements,
while object allocation and modification operations are translated into SQL
\code{INSERT} and \code{UPDATE} statements. To achieve this, the scalar values
retrieved from the database are converted into object instances in the program
and vice versa.

The ORM system in Django maps data models to relational database tables. The
data models are represented as Python classes. The model class also has
attributes, called model fields, which represent the columns of the database
table. These model fields define the data types that are used in the database.
For example, the \code{CharField} model field uses the \code{VARCHAR} data type
in the database. In addition, these model fields also define the behavior of
the columns, such as their \code{NOT NULL} and \code{UNIQUE} constraints in the
database.

Aside from storing and loading data, the ORM system in Django also provides
querying capabilities through the use of lookups and transforms. Lookups and
transforms are specified using keyword arguments in the form of
\code{field\_\_lookuptype=value}, where \code{lookuptype} can be a lookup or
transform name. Lookups define how the \code{WHERE} clause of an SQL query is
composed by Django. Transforms are used to transform a model field from one
form to another (e.g. extracting the year from a \code{DateField}). For
example, a query specified in Python as shown in \autoref{code:query1py} will
be translated into an SQL query equivalent shown in \autoref{code:query1sql}.

\lstinputlisting[language=Python, caption={A query on a \code{DateField} named
	\code{pub\_date} with the \code{year} transform and \code{gte} lookup.},
	label=code:query1py]{codes/2-query1.py}

\lstinputlisting[language=SQL, caption={An SQL query equivalent to
	\autoref{code:query1py}.}, label=code:query1sql]{codes/2-query1.sql}

Django officially supports five database backends: PostgreSQL, MariaDB, MySQL,
SQLite, and Oracle Database \cite{django:databases}. Each database backend
(other than SQLite) requires a compatible database driver to be installed. The
database backends provided by Django adapt the database drivers so that they
can be used with the ORM system. These backends are implemented in the
\code{django.db.backends} module.

%-----------------------------------------------------------------------------%
\section{JSON on Relational Database Systems Supported by Django}
%-----------------------------------------------------------------------------%

JavaScript Object Notation (JSON) is a lightweight, text-based data-interchange
format \cite{json}. It is designed to be human-readable and easy for machines
to parse and generate. It is based on a subset of the JavaScript standard, but
it is completely language independent. It uses conventions that are similar to
popular programming languages such as C, C++, C\#, Java, JavaScript, Python,
and many others. These properties make JSON a widely-used data-interchange
format, especially in web applications.

JSON is built on two structures: an unordered set of key-value pairs (called
JSON objects) and an ordered list of values (called JSON arrays). The values
can be scalar values such as strings, numbers, booleans, or null. However, they
can also be JSON objects or JSON arrays, which means that they can be nested to
form more complex data structures. An example of a JSON object that contains a
JSON array and other JSON objects can be seen in \autoref{code:json}.

\lstinputlisting[language=Python, caption=A JSON object that contains a JSON
	array and other JSON objects., label=code:json]{codes/2-json.json}

To utilize the flexibility of JSON, some relational databases have implemented
support for storing and querying JSON data. This support is commonly
implemented by providing database functions to operate on JSON data stored as
text or, in some cases, a native JSON data type. JSON support on the database
systems supported by Django varies between one another. This section explains
how those database systems provide JSON support.

\subsection{PostgreSQL}

PostgreSQL supports a native \code{JSON} data type and JSON functions as of
version 9.2 \cite{postgresql:9.2}. Data stored using the \code{JSON} data type
is stored as text. However, it has the advantage of validating that each stored
value is valid JSON. The two JSON functions included in the 9.2 release allow
users to convert a row or array in the database into \code{JSON}.

As of version 9.4, PostgreSQL also supports a \code{JSONB} data type and more
JSON functions \cite{postgresql:9.4}. The \code{JSONB} data type supports fast
data retrievals and simple expression search queries using Generalized Inverted
Indexes (GIN). Data stored using the \code{JSONB} data type is stored as
binary. This makes storing data with the \code{JSONB} data type slightly slower
compared to the \code{JSON} data type as it needs to be encoded into binary
prior to saving. However, it is significantly faster to process, since the
processing functions do not need to reparse the data on each execution. The
JSON functions introduced in the 9.4 release enable users to extract and
manipulate JSON data with a performance that matches or surpasses
document-based databases.

\subsection{MariaDB and MySQL}

MariaDB is highly compatible with MySQL, albeit with some behavior differences
\cite{mariadb:compatibility}. This is because MariaDB is originally designed to
be a completely open-source drop-in replacement for MySQL. MariaDB's client
protocol is binary compatible with MySQL's client protocol, therefore Django
uses the same database backend for MariaDB and MySQL \cite{django:databases}.
Both database systems have compatible JSON functions to operate on JSON data,
but the data is stored differently.

MySQL supports a native \code{JSON} data type as of version 5.7.8
\cite{mysql:json}. The \code{JSON} data type provides automatic validation of
JSON data. It is internally stored as binary, which permits quick read access
as the value does not need to be parsed from a text representation. Data stored
using the \code{JSON} data type can be indexed by using a generated column that
extracts a scalar value from the JSON data.

MariaDB has a \code{JSON} data type as of version 10.2.7 \cite{mariadb:json}.
The \code{JSON} data type was introduced for compatibility reasons with MySQL's
\code{JSON} data type. However, the \code{JSON} data type is just an alias for
the \code{LONGTEXT} data type. Despite the data is stored as a string rather
than binary, it can be validated using the provided \code{JSON\_VALID} function
as a \code{CHECK} constraint.

\subsection{SQLite}

SQLite has the JSON1 extension as of version 3.9.0 \cite{sqlite:3.9.0}. The
JSON1 extension is a loadable extension that includes functions that can be
used to manage JSON content \cite{sqlite:json1}. As of this writing, the JSON1
extension stores JSON data as text. The JSON1 extension includes a
\code{JSON\_VALID} function that can be used as a \code{CHECK} constraint for
JSON data.

Some JSON functions provided in the JSON1 extension are similar to those found
in MariaDB and MySQL, but they have different behaviors on SQLite. For example,
the \code{JSON\_EXTRACT} function that can be used to extract JSON values at a
given path is available on MySQL, MariaDB, and SQLite. However, on MariaDB and
MySQL, the function returns string values with the JSON double quotes
(\code{""}) intact, whereas on SQLite, the function returns strings with the
double quotes removed. The function also behaves differently when extracting
the JSON \code{null} value: it returns the SQL string \code{'null'} on
\mbox{MariaDB} and MySQL, but it returns SQL \code{NULL} on SQLite. In
addition, calling the \code{JSON\_VALID} function with SQL \code{NULL} as its
argument returns SQL \code{NULL} on MariaDB and MySQL, but it returns \code{0}
(false) on SQLite.

\subsection{Oracle Database}

Oracle Database supports storing, validating, querying, and indexing JSON data
as of version 12.1.0.2 \cite{oracle:12.1.0.2}. JSON data is stored using the
\code{VARCHAR2}, \code{CLOB}, and \code{BLOB} data types \cite{oracle:json}.
Unlike the other database systems, Oracle Database does not support storing
scalar values in JSON columns, which means that the top-level values in JSON
columns are either JSON objects or JSON arrays. JSON validation can be enforced
using the \code{IS JSON} condition as a \code{CHECK} constraint. Querying JSON
data can be performed by utilizing the provided JSON functions. Indexing JSON
data can be implemented using bitmap, function-based, or composite B-tree
indexes.

As with the other database systems, Oracle Database has functions to operate on
JSON data, but they are named differently. For example, instead of
\code{JSON\_EXTRACT}, Oracle Database has \code{JSON\_VALUE} for extracting
scalar values and \code{JSON\_QUERY} for extracting JSON objects and arrays. In
addition, it has the \code{JSON\_EXISTS} function that is similar to the
\code{JSON\_CONTAINS\_PATH} function on MariaDB and MySQL.

%-----------------------------------------------------------------------------%
\section{\code{JSONField}}
%-----------------------------------------------------------------------------%

In Django, \code{JSONField} is a field for storing or accepting JSON-encoded
data. There are two kinds of \code{JSONField} in Django: the model field and
the form field. The model field lets Django store JSON data to the database,
while the form field lets Django process user input in the form of JSON data
\cite{django30_modeljsonfield, django30_formjsonfield}. In Django version
1.9 to 3.0, \code{JSONField} was only available in the
\code{django.contrib.postgres} module, which means that it could only be used
with the PostgreSQL database backend. Since all of the database systems
supported by Django now have support for managing JSON data, \code{JSONField}
can be implemented for the other database backends.

In addition to storing or accepting JSON-encoded data, the previous
implementation of \code{JSONField} also included \code{JSONField}-specific
lookups and transforms that can be used to query JSON data
\cite{django30_modeljsonfield}. The lookups consisted of the containment
lookups and the key existence lookups. The containment lookups let the users
query for supersets or the subsets of a JSON value. The key existence lookups
let the users query for JSON objects or arrays that have certain keys at a
given path. The transforms let the users query for JSON objects that have
certain values at a given path, by chaining the keys or indexes as the
transform names to compose a JSON path. These lookups and transforms can be
implemented by utilizing the JSON functions provided by the database systems.
