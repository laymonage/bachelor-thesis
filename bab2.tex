%-----------------------------------------------------------------------------%
\chapter{\babDua}
%-----------------------------------------------------------------------------%

This chapter focuses on building a conceptual basis by reviewing existing
literature and documentation of the underlying concepts of this research
on the Django web framework and other supporting technologies.

%-----------------------------------------------------------------------------%
\section{Web Framework and Django}
%-----------------------------------------------------------------------------%

A web framework is a software framework that is designed to support the
development of web applications. It consists of reusable components that solve
common problems in web application development. Most web frameworks provide
safe defaults to prevent security problems in the web application. This
convenience helps programmers build web applications faster and safer. Some web
frameworks also help maintain better application structure by enforcing
software design patterns.

The common components of a web framework are the dispatcher, the decoder, the
generator, and the store \cite{schwarz_webframework}. The dispatcher maps
Uniform Resource Locators (URLs) to application code that is invoked when an
HTTP request is received. The decoder decodes the client request, which allows
the web application to read parameters, headers, and request body data sent as
part of the request. The generator constructs the output that is sent as a
response to the client. The store holds inter-request data, which is usually
stored in a database system.

One of the most popular web frameworks is Django, a free and open-source web
framework written in Python \cite{django}. Its overall philosophies are loose
coupling, less code, quick development, don't repeat yourself (DRY), explicit
is better than implicit, and consistency \cite{django:philosophies}. Django
follows what it calls the model-template-view architectural pattern, which
shares similarities to the model-view-controller pattern \cite{django:faq}. It
consists of an object-relational mapping (ORM) system that handles the
interaction with database systems, a template system that defines how the data
looks like to the user, and a view system that defines which data is presented
to the user. Among all of those features in Django, the ORM system is the most
complex and is central to this research.

%-----------------------------------------------------------------------------%
\section{Object-Relational Mapping (ORM) in Django}
%-----------------------------------------------------------------------------%

Object-relational mapping (ORM) is a programming technique that allows its
users to query and manipulate data stored in a relational database using an
object-oriented paradigm \cite{linskey:orm}. Object query, load, and delete
operations are translated into SQL \verb|SELECT| and \verb|DELETE| statements,
while object allocation and modification operations are translated into SQL
\verb|INSERT| and \verb|UPDATE| statements. To create this mapping, scalar
values retrieved from the database are converted into object instances in the
program and vice versa.

The ORM system in Django maps data models to relational database tables. The
data models are represented as Python classes. A model class has attributes,
called model fields, which represent the columns of the database table. These
model fields define the data types and constraints that are used in the
database.

\listing
{Python}
{A \code{Person} model.}
{code:personmodel}
{codes/2-personmodel.py}

As an example, to store user information, a \verb|Person| model as shown by
\autoref{code:personmodel} can be created. The \verb|Person| model has a
\verb|name| attribute as a \verb|CharField| to store the name of the user. The
model also has a \verb|dob| attribute as a \verb|DateField| to store the user's
date of birth.

\begin{table}
	\centering
	\texttt{\begin{tabular}{|l|l|l|}
		\multicolumn{1}{c}{\textbf{myapp\_person}} \\ \hline
		id & name          & dob        \\ \hline
		42 & Sage Abdullah & 1999-06-09 \\ \hline
		21 & Alice Doe     & NULL       \\ \hline
	\end{tabular}}
	\caption{A relational database table \code{myapp\_person} represented by
	the \code{Person} model in \autoref{code:personmodel}.}
	\label{table:myapp_person}
\end{table}

The \verb|Person| model shown by \autoref{code:personmodel} represents the
relational database table shown by \autoref{table:myapp_person}. The table name
is derived from the Django application name (in this case \verb|myapp|) and the
model name. The table schema is derived from the model fields: the \verb|name|
field corresponds to the \verb|name| column with the SQL \verb|VARCHAR| data
type, while the \verb|dob| field corresponds to the \verb|dob| column with the
SQL \verb|DATE| data type and without the \verb|NOT NULL| constraint.

\listing
{Python}
{A query on a \code{DateField} named \code{pub\_date} with the \code{year}
transform and \code{gte} lookup.}
{code:query1py}
{codes/2-query1.py}

\listing
[0 pt]
{SQL}
{An SQL query equivalent to \autoref{code:query1py}.}
{code:query1sql}
{codes/2-query1.sql}

Aside from storing and loading data, the ORM system in Django also provides
querying capabilities through the use of lookups and transforms. Lookups and
transforms are specified using keyword arguments in the form of
\verb|field__lookuptype=value|, where \verb|lookuptype| can be a lookup or
transform name. Lookups define how the \verb|WHERE| clause of an SQL query is
composed by Django. Transforms are used to transform a model field from one
form to another (e.g. extracting the year from a \verb|DateField|). For
example, a query specified in Python as shown by \autoref{code:query1py} will
be translated into an SQL query equivalent shown by \autoref{code:query1sql}.

Django officially supports five database backends: PostgreSQL, MariaDB, MySQL,
SQLite, and Oracle Database \cite{django:databases}. Each database backend
requires a compatible database driver to be installed. The database backends
provided by Django adapt the database drivers so that they can be used with the
ORM system. These backends are implemented in the \verb|django.db.backends|
module. To implement \verb|JSONField| that works on all database backends,
further study of the database systems is required.

%-----------------------------------------------------------------------------%
\section{JSON on Relational Database Systems Supported by Django}
%-----------------------------------------------------------------------------%

JavaScript Object Notation (JSON) is a lightweight, text-based data-interchange
format \cite{json}. It is designed to be human-readable and easy for machines
to parse and generate. It is based on a subset of the JavaScript standard, but
it is completely language independent. It uses conventions that are similar to
popular programming languages such as C, C++, C\#, Java, JavaScript, Python,
and many others. These properties make JSON a widely-used data-interchange
format, especially in web applications.

\noindent
\begin{minipage}{\linewidth}
\lstinputlisting[language=Python, caption=A JSON object.,
label=code:json]{codes/2-json.json}
\end{minipage}

JSON is built on two structures: an unordered set of key-value pairs (called
JSON objects) and an ordered list of values (called JSON arrays)
\cite{json:org}. The values can be scalar values such as strings, numbers,
booleans, or \verb|null|. However, they can also be JSON objects or JSON
arrays, which means that they can be nested to form more complex data
structures. An example of a JSON object that contains a JSON array and other
JSON objects can be seen in \autoref{code:json}.

To specify the value at a given path in some JSON data, the JSONPath notation
is commonly used. The JSONPath notation has no formal specification, but the
creator has written an article that describes the notation
\cite{goessner:jsonpath}. The notation is based on the XPath notation for XML.
The basic concept of JSONPath is the use of \verb|$| to indicate the root
object/element, \verb|.key| or \verb|['key']| to select a key, and \verb|[i]|
to select the element at index \verb|i| of an array. For example, using a
JSONPath like \verb|$.phoneNumbers[0].number| on the JSON object shown by
\autoref{code:json} will select the string \verb|"001 123 4567"|.

As seen in the example shown by \autoref{code:json}, the schema of a JSON
object is defined within the data structure itself, which allows each object to
define its own schema for flexibility. To utilize the flexibility of JSON, some
relational databases have implemented support for storing and querying JSON
data. This support is commonly implemented by providing database functions to
operate on JSON data stored as text or, in some cases, a native JSON data type.
JSON support on the database systems supported by Django varies between one
another. The following subsections explain how those database systems provide
JSON support.

\subsection{PostgreSQL}

PostgreSQL supports a native \verb|json| data type and JSON functions as of
version 9.2 \cite{postgresql:9.2}. Data stored using the \verb|json| data type
is stored as text. However, it has the advantage of validating that each stored
value is valid JSON. The two JSON functions included in version 9.2 allow the
users to convert a row or array in the database into \verb|json|.

As of version 9.4, PostgreSQL also supports a \verb|jsonb| data type and more
JSON functions \cite{postgresql:9.4}. The \verb|jsonb| data type supports fast
data retrievals and simple expression search queries using Generalized Inverted
Indexes (GIN). Data stored using the \verb|jsonb| data type is stored as
binary. This makes storing data with the \verb|jsonb| data type slightly slower
compared to the \verb|json| data type as it needs to be encoded into binary
prior to saving. However, it is significantly faster to process, since the
processing functions do not need to reparse the data on each execution. The
JSON functions introduced in the 9.4 release enable users to extract and
manipulate JSON data with a performance that matches or surpasses
document-based databases.

For the PostgreSQL database backend, Django uses the Psycopg 2 database driver.
Psycopg 2 supports the \verb|json| and \verb|jsonb| data types as of version
2.5.4. It offers a JSON adapter that serializes Python objects before they are
sent to the database. By default, it also automatically deserializes JSON data
from the database into Python objects. Psycopg 2 uses Python's built-in JSON
library to perform the serialization and deserialization
\cite{psycopg2:json-adaptation}.

\subsection{MariaDB and MySQL}

MariaDB is highly compatible with MySQL, albeit with some behavior differences
\cite{mariadb:compatibility}. They are compatible because MariaDB was
originally designed to be a completely open-source drop-in replacement for
MySQL. MariaDB's client protocol is binary compatible with MySQL's client
protocol, hence Django uses the same database backend for MariaDB and MySQL
\cite{django:databases}. Despite the similarities, their behaviors are
different when operating on JSON data.

MySQL supports a native \verb|json| data type as of version 5.7.8
\cite{mysql:json}. The \verb|json| data type provides automatic validation of
JSON data. It is internally stored as binary, which permits quick read access
as the value does not need to be parsed from a text representation. Data stored
using the \verb|json| data type can be indexed by using a generated column that
extracts a scalar value from the JSON data.

MariaDB has a \verb|json| data type as of version 10.2.7 \cite{mariadb:json}.
The \verb|json| data type was introduced for compatibility reasons with MySQL's
\verb|json| data type. However, on MariaDB, the \verb|json| data type is just
an alias for the \verb|longtext| data type. Even though the data is stored as a
string rather than binary, it can be validated using the provided
\verb|JSON_VALID()| function as a \verb|CHECK| constraint.

\subsection{SQLite}

SQLite has the JSON1 extension as of version 3.9.0 \cite{sqlite:3.9.0}. The
JSON1 extension is a loadable extension that includes functions that can be
used to manage JSON content \cite{sqlite:json1}. As of this writing, the JSON1
extension stores JSON data as \verb|text|. The JSON1 extension includes a
\verb|JSON_VALID()| function that can be used as a \verb|CHECK| constraint for
JSON data.

Some JSON functions provided in the JSON1 extension are similar to those found
in MariaDB and MySQL, but they have different behaviors on SQLite. For example,
the \verb|JSON_EXTRACT()| function that can be used to extract JSON values at a
given path is available on MySQL, MariaDB, and SQLite. However, on MariaDB and
MySQL, the function returns string values with the JSON double quotes
(\verb|""|) intact, whereas on SQLite, the function returns strings with the
double quotes removed. The function also behaves differently when extracting
the JSON \verb|null| value: it returns the SQL string \verb|'null'| on MariaDB
and MySQL, but it returns SQL \verb|NULL| on SQLite. In addition, calling the
\verb|JSON_VALID()| function with SQL \verb|NULL| as its argument returns SQL
\verb|NULL| on MariaDB and MySQL, but it returns \verb|0| (false) on SQLite.

\subsection{Oracle Database}

Oracle Database supports storing, validating, querying, and indexing JSON data
as of version 12.1.0.2 \cite{oracle:12.1.0.2}. JSON data is stored using the
\verb|VARCHAR2| and \verb|LOB| data types \cite{oracle:json}. Unlike the other
database systems, Oracle Database does not support storing scalar values in
JSON columns, which means that the top-level values in JSON columns are either
JSON objects or JSON arrays. JSON syntax validation can be enforced using the
\verb|IS JSON| condition as a \verb|CHECK| constraint. Querying JSON data can
be performed by utilizing the provided JSON functions. Indexing JSON data can
be implemented using bitmap, function-based, or composite B-tree indexes.

As with the other database systems, Oracle Database has functions to operate on
JSON data, but they are named differently. For example, instead of
\verb|JSON_EXTRACT()|, Oracle Database has \verb|JSON_VALUE()| for extracting
scalar values and \verb|JSON_QUERY()| for extracting JSON objects and arrays.
In addition, it has the \verb|JSON_EXISTS()| function that is similar to the
\verb|JSON_CONTAINS_PATH()| function on MariaDB and MySQL.

The previous paragraphs and subsections have explained how each database
system supported by Django manages JSON data in its own ways. Meanwhile, the
implementation of \verb|JSONField| has to provide the same programming
interface regardless which database backend is used. Therefore, it is important
to see how the programming interface for \verb|JSONField| can be used by Django
programmers.

%-----------------------------------------------------------------------------%
\section{\code{JSONField}}
%-----------------------------------------------------------------------------%

Django provides a variety of model fields and form fields to be used with its
ORM and forms features. Each field defines its own data type and behavior to
manage the data. There are fields that store simple data such as
\verb|IntegerField| and \verb|TextField|, but there are also those that store
fairly complex data such as \verb|FileField| and \verb|ImageField|.

In Django, \verb|JSONField| is a field for storing or accepting JSON-encoded
data. There are two kinds of \verb|JSONField| in Django: the model field and
the form field. The model field lets Django store JSON data to the database,
while the form field lets Django process user input in the form of JSON data
\cite{django30_modeljsonfield, django30_formjsonfield}. In Django version
1.9 to 3.0, \verb|JSONField| was only available in the
\code{django.contrib.postgres} module, which means that it could only be used
with the PostgreSQL database backend.

\listing
{Python}
{A Django model named \code{Profile} that stores a user's configurations in a
\code{JSONField}.}
{code:profilemodel}
{codes/2-profilemodel.py}

\listing
{Python}
{A Python shell session that demonstrates how storing, modifying, and
retrieving data in a \code{JSONField} works.}
{code:jsonshell}
{codes/2-shell.py}

To use the \verb|JSONField| model field, the programmer can define a Django
model and declare a field with an instance of \verb|JSONField| as demonstrated
by \autoref{code:profilemodel}. After that, the field can be used as
demonstrated by \autoref{code:jsonshell}. Lines 2 and 3 of the listing
demonstrate that the field accepts a Python object that can be encoded into
JSON, in this case a dictionary. When calling the \verb|create()| method, the
object inside the field gets encoded before it is saved to the database. Upon
retrieval, the data gets decoded back into a Python object, as demonstrated by
lines 5 to 9. Since the data is a Python dictionary, modifying the data using
dictionary operations is possible, as shown by line 10. After saving the
modified model object and retrieving it again, it can be verified that the
modification was successfully stored into the database, as shown by lines 11 to
13.

\listing
{Python}
{A Python shell session that demonstrates how \code{JSONField} lookups and
transforms work.}
{code:shelllookups}
{codes/2-shelllookups.py}

In addition to storing or accepting JSON-encoded data, \verb|JSONField| also
includes \verb|JSONField|-specific lookups and transforms that can be used to
query JSON data, as shown by \autoref{code:shelllookups}
\cite{django30_modeljsonfield}. The lookups consist of the containment
lookups and the key existence lookups. The containment lookups let the users
query for supersets or the subsets of a JSON value, as shown by lines 8 to 10
of the listing. The key existence lookups let the users query for JSON objects
or arrays that have certain keys at a given path, as shown by lines 11 and 12.
The transforms let the users query for JSON values at a given path by chaining
the keys or indexes as the transform names to compose a JSON path, as shown by
lines 13 to 16. These lookups and transforms can be implemented by utilizing
the JSON functions provided by the database systems.

\listing
{Python}
{An excerpt of the PostgreSQL-only \code{JSONField} implementation.}
{code:pgjsonfield}
{codes/2-jsonfield.py}

Before this research, the existing implementation of \verb|JSONField| relied on
features that are only available on PostgreSQL \cite{django:pgjsonfield}. For
example, it used the \verb|Json| adapter included in Psycopg 2 (the database
driver for PostgreSQL), as shown by lines 2 and 5 of
\autoref{code:pgjsonfield}. The adapter was used to encode a Python object into
a JSON-encoded string before it is saved to the database (implemented in the
\verb|get_prep_value()| method). In addition, it used the \verb|jsonb| data
type that is available on PostgreSQL, but not on other database systems
supported by Django.

In order to reimplement \verb|JSONField| for all database backends, the
implementation design has to take into account the behavior differences of the
database systems and compatibility with the existing implementation.
This chapter has explained how the ORM in Django works, how JSON is supported
on the database systems, and how the existing \verb|JSONField| can be used by
Django developers. With this information, a new implementation of
\verb|JSONField| for all database backends can be designed.
