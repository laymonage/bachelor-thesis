%
% Halaman Abstract
%
% @author  Andreas Febrian
% @version 1.00
%

\chapter*{Abstract}
\singlespacing

\vspace*{0.2cm}

\noindent \begin{tabular}{l l p{11.0cm}}
	Name&: & \penulis \\
	Program&: & \program \\
	Title&: & \judulInggris \\
\end{tabular} \\

\vspace*{0.5cm}

\noindent
This thesis explains the implementation and analysis of JSONField in the Django
web framework. JSONField allows the user to store and query semi-structured
data in a relational database through the use of Django's object-relational
mapper (ORM). The implementation has to account for the compatibility with all
the database systems officially supported by Django, namely PostgreSQL, SQLite,
MySQL, MariaDB, and Oracle Database. The implementation was created as part of
the Google Summer of Code 2019 program and was merged into the Django codebase
for the release of Django 3.1. In addition, this thesis also covers an analysis
of JSONField usage with Django's built-in model validation feature to validate
the JSON data, as well as a benchmark analysis of JSONField usage with a JSON
testbed.\\

\vspace*{0.2cm}

\noindent \bo{Keywords:} \\
JSONField, JSON, Django, database, object-relational mapper, semi-structured
data \\

\onehalfspacing
\newpage
