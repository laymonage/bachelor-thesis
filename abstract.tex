%
% Halaman Abstract
%
% @author  Andreas Febrian
% @version 1.00
%

\chapter*{Abstract}
\singlespacing

\vspace*{0.2cm}

\noindent \begin{tabular}{@{}l l p{12.0cm}}
	Name&: & \penulis \\
	Program&: & \program \\
	Title&: & \judulInggris \\
\end{tabular} \\

\vspace*{0.5cm}

\noindent
In web development, it is a common practice to use a web framework to build web
applications. One of the most popular web frameworks is Django, a free and
open-source web framework written in Python. Among the wide range of features
in Django, the object-relational mapping (ORM) system is the most complex. The
ORM system in Django maps data models to relational database tables. The data
models are defined as Python classes that have attributes known as model
fields. One of the model fields available in Django is \verb|JSONField| that
allows programmers to store and query semi-structured data using the JSON data
format in a relational database. Before this research, \verb|JSONField| was
only available for the PostgreSQL database system. Meanwhile, Django officially
supports PostgreSQL, MariaDB, MySQL, SQLite, and Oracle Database. This research
aimed to implement a new \verb|JSONField| that is compatible with all database
systems supported by Django. In addition to the implementation, this research
also covers some examples of \verb|JSONField| usage with Django's built-in
model validation feature to validate JSON data in a \verb|JSONField|.
The process for implementing \verb|JSONField| includes researching JSON data
support on the database systems supported by Django, designing, and
implementing \verb|JSONField|. The \verb|JSONField| implementation is tested on
all database systems using automated tests. This research would hopefully
provide insights to Django users about the inner workings and validation
examples of \verb|JSONField|.\\

\vspace*{0.2cm}

\noindent \bo{Keywords:} \\
\verb|JSONField|, JSON, Django, Database, Object-relational mapping,
Semi-structured data \\

\onehalfspacing
\newpage
