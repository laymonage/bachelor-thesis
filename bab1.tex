%-----------------------------------------------------------------------------%
\chapter{\babSatu}
%-----------------------------------------------------------------------------%

This chapter explains the background of this research and the problem to be
solved by this research.

%-----------------------------------------------------------------------------%
\section{Research Background}
%-----------------------------------------------------------------------------%

In the early days of web development, every part of a web application was coded
manually. This required extensive knowledge of how the web works. Depending on
the web application, this might include understanding of the Hypertext Transfer
Protocol (HTTP), the Hypertext Markup Language (HTML), and other things such as
databases. If not done correctly, this could lead to a lot of errors and
security holes in the web application.

Web frameworks aim to solve this problem by providing a standard way to build
web applications. Web frameworks help eliminate the overhead of implementing
common parts of web applications such as the HTTP layer, the database layer,
and the view layer. This helps web developers build web applications more
quickly and safely by focusing more on the business logic of the web
application itself instead of the more intricate details.

Among the widely-used web frameworks is Django, a free and open-source web
framework written in Python. One of the main features of Django is an
object-relational mapper (ORM) in the database layer. The ORM maps the data
models, represented as Python classes, to tables in a relational database. The
columns of a table are defined as attributes (called fields) inside the model
class.

While relational databases are useful in handling structured data, there has
been an increase in the popularity of non-relational databases, commonly known
as NoSQL databases. NoSQL encourages the simplicity of database design, which
makes the process of scaling the database to multiple clusters of machines
easier. NoSQL databases are also considered as more flexible due to its dynamic
schema.

However, NoSQL databases also have their own downsides. Most NoSQL databases
disregard the Atomicity, Consistency, Isolation, and Durability (ACID)
principles commonly found in relational database transactions. They also lack
the ability to perform joins across tables, which makes it harder to deal with
entity relations.

Nowadays, some relational database systems have come up with their own
solutions of providing a hybrid data model. This is done by combining
structured data with some form of semi-structured data, most commonly in
JavaScript Object Notation (JSON) format. This gives all the benefits of using
a relational database, while still allowing the simplicity and dynamic nature
of semi-structured data.

From version 1.9 until version 3.0, Django had provided an implementation of
JSONField. However, it was exclusively available for PostgreSQL. This JSONField
allowed its users to store and query JSON data in a relational database column
using the jsonb data type found only on PostgreSQL at the time.

Django officially supports PostgreSQL, SQLite, MySQL, MariaDB, and Oracle
Database backends. Over the years since the initial PostgreSQL JSONField
implementation, those database systems have developed their own support for
dealing with JSON data. However, Django had yet to implement JSONField for
database backends other than PostgreSQL.

The limited support for JSONField prompted the Django community to create their
own JSONField implementations for other database backends. Some of them target
one specific database backend and utilize various functions provided by the
database system to add extended querying capabilities. Some other
implementations focus on cross-database support, so they are only implemented
as text-based fields that do not have any extended querying capabilities.

The individual efforts for implementing JSONField results in the abundance of
third-party JSONField packages on the Python Package Index. Some of them are
very popular, such as the jsonfield package that has more than 1100 stars on
GitHub. This indicates a popular demand for JSONField. However, it also
indicates a fragmentation problem as people use different JSONField packages.
Thus, there's a motivation to bring JSONField into Django's core model fields,
with cross-database and extended querying support in mind, as well as
compatibility with the previous implementation of JSONField.

%-----------------------------------------------------------------------------%
\section{Permasalahan}
%-----------------------------------------------------------------------------%
\todo{Sebutkan permasalahan penelitian Anda dari latar belakang tersebut.}

%-----------------------------------------------------------------------------%
\subsection{Definisi Permasalahan}
%-----------------------------------------------------------------------------%
Berikut ini adalah rumusan permasalahan dari penelitian yang dilakukan:
\begin{itemize}
	\item Bagaimana cara membuat pertanyaan penelitian?
\end{itemize}
\todo{Tuliskan permasalahan yang ingin diselesaikan. Bisa juga berbentuk pertanyaan}


%-----------------------------------------------------------------------------%
\subsection{Batasan Permasalahan}
%-----------------------------------------------------------------------------%
Berikut ini adalah asumsi yang digunakan sebagai batasan penelitian ini:
\begin{itemize}
	\item Salah satu batasannya adalah, ini hanya \f{template}.
\end{itemize}
\todo{Umumnya ada asumsi atau batasan yang digunakan untuk menjawab
pertanyaan-pertanyaan penelitian diatas.}


%-----------------------------------------------------------------------------%
\section{Tujuan Penelitian}
%-----------------------------------------------------------------------------%
Berikut ini adalah tujuan penelitian yang dilakukan:
\begin{itemize}
	\item Untuk memberikan \f{template} yang dapat mempermudah skripsi orang
	lain.
\end{itemize}
\todo{Tuliskan tujuan penelitian Anda di bagian ini.}


%-----------------------------------------------------------------------------%
\section{Posisi Penelitian}
%-----------------------------------------------------------------------------%
\todo{Sebutkan posisi penelitian Anda. Ada baiknya jika Anda menggunakan gambar
atau diagram. Template ini telah menyediakan contoh cara memasukkan gambar.}

\begin{figure}
	\centering
	\includegraphics[width=0.4\textwidth]{pics/makara.png}
	\caption{Penjelasan singkat terkait gambar.}
	\label{fig:research_position}
\end{figure}

\todo{Jelaskan \pic~\ref{fig:research_position} di sini.}


%-----------------------------------------------------------------------------%
\section{Langkah Penelitian}
%-----------------------------------------------------------------------------%
Berikut ini adalah langkah penelitian yang telah dilakukan:
\begin{enumerate}
	\item Tinjauan literatur \\
	Pada tahap ini, dipelajari teori-teori yang terkait dengan penelitian ini
	untuk mendapatkan konsep dasar yang dibutuhkan dalam mencapai tujuan
	penelitian.
	\item Analisis implementasi dan kesimpulan \\
	Pada tahap ini, digunakan studi kasus untuk analisis terkait kegunaan
	\f{template}. Setelah melakukan analisis tersebut, ditarik kesimpulan
	keseluruhan dari penelitian ini.
\end{enumerate}


%-----------------------------------------------------------------------------%
\section{Sistematika Penulisan}
%-----------------------------------------------------------------------------%
Sistematika penulisan laporan adalah sebagai berikut:
\begin{itemize}
	\item Bab 1 \babSatu \\
	    Bab ini mencakup latar belakang, cakupan penelitian, dan pendefinisian
	    masalah.
	\item Bab 2 \babDua \\
	    Bab ini mencakup pemaparan terminologi dan teori yang terkait dengan
	    penelitian berdasarkan hasil tinjauan pustaka yang telah digunakan,
	    sekaligus memperlihatkan kaitan teori dengan penelitian.
	\item Bab 3 \babTiga \\
	    Apa itu Bab 3?
	\item Bab 4 \babEmpat \\
		Apa itu Bab 4?
	\item Bab 5 \babLima \\
	    Apa itu Bab 5?
	\item Bab 6 \kesimpulan \\
	    Bab ini mencakup kesimpulan akhir penelitian dan saran untuk
	    pengembangan berikutnya.
\end{itemize}

\todo{Anda bisa mengubah atau menambahkan penjelasan singkat mengenai isi
masing-masing bab. Setiap tugas akhir pasti ada yang berbeda pada bagian ini.}
