%---------------------------------------------------------------
\chapter{\kesimpulan}
%---------------------------------------------------------------
This chapter discusses the conclusions of this research and suggestions for
future research.

%---------------------------------------------------------------
\section{Conclusions}
%---------------------------------------------------------------

The primary objective of this research is to implement a new \verb|JSONField|
that can be used on all database backends supported by Django. The
\verb|JSONField| implementation includes the model field, form field, lookups,
and transforms. The implementation should handle the differences in how JSON
data is managed by all of the database systems, which consist of PostgreSQL,
MariaDB, MySQL, SQLite, and Oracle Database.

To handle the differences between the database systems, some adjustments to the
database backends were needed. The adjustments included the additions of new
feature flags in the \verb|DatabaseFeature| classes of the database backends.
The flags indicate the different behaviors of each database system, which can
be used to determine how the implementation on the database system should be
created. Some adjustments were also made to the \verb|DatabaseWrapper| classes
to define the data types and SQL \verb|CHECK| constraints on the database
level. Some database backends also require some adjustments to the
\verb|DatabaseOperations| class to define some casting operations for JSON
data.

After creating the database backend adjustments, \verb|JSONField| and its
lookups and transforms could be implemented. The \verb|JSONField| model field
was implemented by serializing the data before it is sent to the database and
deserializing the data when it is retrieved from the database. On the other
hand, the \verb|JSONField| form field could be used without any modifications
as it does not interact with the database directly. Both the model field and
form fields were updated to support custom encoder and decoder for
serialization and deserialization. The lookups and transforms were implemented
by utilizing the JSON functions that are available on the database systems.

The secondary objective of this research is to provide examples of how
validation rules can be applied to a \verb|JSONField|. For demonstration, a
small Django project with an e-commerce scenario was created. The project
includes two models, \verb|Product| and \verb|Cart|, that represent a product
and a shopping cart on an e-commerce website. The \verb|Cart| model has a
\verb|JSONField| that uses validators to validate the data. The validators were
added using the validation feature in Django. By utilizing the validators, JSON
data in a model field could be validated by calling the \verb|clean_fields()|
method of the model instance and catching the \verb|ValidationError| exception.

To ensure that the new implementation of \verb|JSONField| is consistent on all
database backends, \verb|JSONField| tests were run on all database systems. The
tests included test cases for the new features and documented behaviors, such
as the custom decoder support and the \verb|JSONField| behavior when handling
JSON \verb|null| and SQL \verb|NULL| values. In addition, the \verb|JSONField|
tests also incorporated the tests from the previous PostgreSQL-only
implementation to ensure that backward compatibility is preserved. After
passing the tests, the previous implementation was deprecated by creating
replicas of the classes and raising system warnings about the deprecation.

%---------------------------------------------------------------
\section{Suggestions}
%---------------------------------------------------------------

During the development of this research, some tools were very helpful in aiding
the implementation process. These tools may also be helpful for those who want
to continue improving the Django web framework or those who maintain
third-party packages for Django. Some of the tools in particular are as follow:

\begin{enumerate}
	\item  For running the Django test suite, it is highly recommended to use
	       the django-docker-box
	       tool.\footnote{\url{https://github.com/django/django-docker-box}}
	       The tool makes it easy to run the Django test suite across all
	       supported databases and Python versions by running them inside
	       Docker containers. During the development of this research,
	       django-docker-box proved to be an indispensable tool as it allowed
	       the \verb|JSONField| implementation to be tested quickly on a local
	       machine prior to pushing the changes to the forked Django
	       repository.
	\item  For trying out SQL queries on different database systems, the DB
		   Fiddle tool is
		   recommended.\footnote{\url{https://www.db-fiddle.com}} The tool
		   allows its users to execute SQL queries through the web interface
		   without having to run a database server or client on their machine.
		   When implementing the lookups and transforms for \verb|JSONField|,
		   the tool was very helpful in verifying the behavior of JSON
		   functions on each database system.
	\item  For code style checking, the pre-commit tool is
		   recommended.\footnote{\url{https://pre-commit.com}} The tool can be
		   configured to automatically run the
		   flake8\footnote{\url{https://flake8.pycqa.org}} and
		   isort\footnote{\url{https://pycqa.github.io/isort}} linters (which
		   are used by Django) before creating a Git commit. Thus, it can
		   prevent code style issues that do not follow Django's code style
		   guide from being included in Git commits, simplifying the review
		   process for the Django maintainers. In fact, Django has since
		   integrated the pre-commit tool into the repository at the time of
		   this writing.\footnote{https://github.com/django/django/blob/master/.pre-commit-config.yaml}
\end{enumerate}

In addition, there are some suggestions for future research based on the
results of this research. Some of those suggestions are as follow:

\begin{enumerate}
	\item The Query Expression API in Django supports a variety of use cases
		  through the use of built-in and custom query expressions. While the
		  tests in this research have included the common use cases, it is not
		  feasible to test each and every possible use case with
		  \verb|JSONField|. In addition, the \verb|JSONField| implementation in
		  this research has not yet supported efficient partial updates to JSON
		  data using the \verb|update()| method of the QuerySet API. Thus, more
		  complex use cases of \verb|JSONField| with Django's ORM tool on all
		  database backends may be a subject of future research.
	\item This research has provided some examples of JSON data validation,
		  including a basic example of how the schema of a \verb|JSONField| can
		  be validated. For more complex schemas, third-party packages can be
		  utilized to create the validation rules. Integrating the third-party
		  packages with Django's validators may be explored in future
		  research.
	\item Recent studies have shown interest in measuring the performance of
		  relational databases in handling JSON data. Relational databases are
		  not dedicated for storing JSON data, thus they may not perform well
		  compared to NoSQL databases. By contrast, recent benchmarks have
		  shown that a relational database system could outperform a NoSQL
		  database system in varied workloads with JSON data
		  \cite{enterprisedb_benchmark}. Some extensive benchmarks on multiple
		  database systems have also shown that the performance differences
		  between relational and NoSQL databases were not significant,
		  especially for smaller documents \cite{dolgov_benchmark}. However,
		  these benchmarks were done directly on the database systems instead
		  of using an ORM tool. Therefore, further research on relational
		  databases' performance in handling JSON data using an ORM tool such
		  as \verb|JSONField| would provide new insights on this topic.
\end{enumerate}
