%
% Halaman Abstrak
%
% @author  Andreas Febrian
% @version 1.00
%

\chapter*{Abstrak}
\singlespacing

\vspace*{0.2cm}

\noindent \begin{tabular}{l l p{10cm}}
	Nama&: & \penulis \\
	Program Studi&: & \programIndonesia \\
	Judul&: & \judulIndonesia \\
\end{tabular} \\

\vspace*{0.5cm}

\noindent
Dalam pengembangan web, penggunaan sebuah \f{web framework} merupakan praktik
yang umum untuk membangun aplikasi web. Salah satu \f{web framework} yang
populer adalah Django, sebuah \f{web framework} yang bebas dan bersumber
terbuka yang ditulis dalam bahasa Python. Di antara fitur-fitur yang tersedia
pada Django, sistem \f{object-relational mapping} (ORM) adalah yang paling
kompleks. Sistem ORM pada Django memetakan model data ke tabel dalam basis data
relasional. Model data tersebut didefinisikan sebagai \f{class} dalam Python
yang memiliki atribut yang dinamakan \f{model field}. Salah satu \f{model
field} yang tersedia pada Django adalah \verb|JSONField| yang memungkinkan
pemrogram untuk menyimpan dan mencari data semiterstruktur menggunakan format
data JSON dalam basis data relasional. Sebelum penelitian ini, \verb|JSONField|
hanya tersedia untuk sistem basis data PostgreSQL. Sementara itu, Django secara
resmi mendukung PostgreSQL, MariaDB, MySQL, SQLite, dan Oracle Database.
Penelitian ini bertujuan untuk mengimplementasikan \verb|JSONField| baru yang
kompatibel dengan seluruh sistem basis data yang didukung oleh Django.
Implementasi tersebut dibuat sebagai bagian dari program Google Summer of Code
2019 dan telah digabungkan ke dalam basis kode Django untuk perilisan Django
3.1. Selain dari implementasi tersebut, penelitian ini juga membahas contoh
penggunaan \verb|JSONField| dengan fitur validasi model yang ada pada Django
untuk memvalidasi data JSON. \\

\vspace*{0.2cm}

\noindent \bo{Kata kunci:} \\
\verb|JSONField|, JSON, Django, basis data, \f{object-relational mapping}, data
semi-terstruktur \\

\onehalfspacing
\newpage
