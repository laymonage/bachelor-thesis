%
% Halaman Abstrak
%
% @author  Andreas Febrian
% @version 1.00
%

\chapter*{Abstrak}
\singlespacing

\vspace*{0.2cm}

\noindent \begin{tabular}{l l p{10cm}}
	Nama&: & \penulis \\
	Program Studi&: & \programIndonesia \\
	Judul&: & \judulIndonesia \\
\end{tabular} \\

\vspace*{0.5cm}

\noindent
Skripsi ini menjelaskan tentang implementasi dan analisis JSONField pada
\f{framework} web Django. JSONField memungkinkan penggunanya untuk menyimpan
dan mencari data semi-terstruktur dalam basis data relasional dengan
memanfaatkan alat \f{object-relational mapping} yang ada pada Django.
Implementasi ini harus memperhitungkan kompatibilitas dengan semua sistem basis
data yang didukung secara resmi oleh Django, yakni PostgreSQL, SQLite, MySQL,
MariaDB, dan Oracle Database. Implementasi ini dibuat sebagai bagian dari
program Google Summer of Code 2019 dan telah digabungkan ke dalam basis kode
Django untuk perilisan Django 3.1. Selain itu, skripsi ini juga membahas
analisis penggunaan JSONField dengan fitur validasi model yang ada pada Django
untuk memvalidasi data JSON, serta tolok ukur penggunaan JSONField dengan data
uji. \\

\vspace*{0.2cm}

\noindent \bo{Kata kunci:} \\
JSONField, JSON, Django, basis data, \f{object-relational mapping}, data
semi-terstruktur \\

\onehalfspacing
\newpage
