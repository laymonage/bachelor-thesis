%-----------------------------------------------------------------------------%
\chapter*{\kataPengantar}
%-----------------------------------------------------------------------------%

Praise be to Allah, for the blessing and mercy given to me throughout my study
so that I can finally finish this thesis. My main motivation to write this
thesis originally stemmed from my passion for open-source software development.
Open-source software has allowed me to tremendously learn and grow as a
computer scientist. Through open-source software, I could study how pieces of
software come together and form state-of-the-art tools that are widely used by
other people. It has opened me to various opportunities in life that I
otherwise could not achieve. Therefore, I believe that it is my duty to
contribute back to the open-source community by sharing the work I have done
for one of my favorite open-source software, the Django web framework, in the
form of this thesis. In the making of this thesis, there were great people who
helped me in various ways. Thus, I would like to express my sincere gratitude
to:

\begin{enumerate}
    \item My thesis advisors, \pembimbingSatu\ and \pembimbingDua, for their
          advice, support, and guidance throughout the writing of this thesis.
    \item The Django community, especially my mentors: Carlton Gibson, Mariusz
          Felisiak, Adam Johnson, and Raphael Michel, for all their help and
          support during the development of my Google Summer of Code project
          which served as the groundwork for this thesis.
    \item My beloved parents and my dear sisters. This thesis would not be
          possible without the continuous support and prayers from them. May
          Allah bless them and protect them now and forever.
    \item My academic advisor, Siti Aminah S.Kom., M.Kom., and all the
          lecturers of the Faculty of \fakultas, Universitas Indonesia
          (Fasilkom UI) for the support, knowledge, and wisdom they have given
          me during my study.
    \item My friends at Fasilkom UI, especially Tarung (Fasilkom UI 2017), who
          supported me in my study from start to finish. In particular, I would
          like to thank Shafiya for cheering me up amid the COVID-19 pandemic
          during the final year of my study. I would also like to thank Tan and
          Gio for the many endeavors we went through.
    \item Sayid, Sena, and my future better half, as well as Ato, Juki, Humam,
          and Yoga. They have been some of the few people for me to confide and
          share my life experiences.
    \item Every single person who has given me lessons in life, both directly
          or indirectly, that made me become the person I am today.
\end{enumerate}

I understand that this thesis is not perfect. Therefore, should there be any
criticisms, suggestions, and/or questions regarding this thesis, please send
them directly to me via
\code{\href{mailto:sage.muhammad@ui.ac.id}{sage.muhammad@ui.ac.id}}. I do hope
that this thesis will be useful for anyone who reads it, especially the
students of the Faculty of \fakultas, Universitas Indonesia and the Django
community.

\vspace*{0.1cm}
\begin{flushright}
Depok, \tanggalSiapSidang\\[0.1cm]
\vspace*{1.75cm}
\penulis

\end{flushright}
