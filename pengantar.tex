%-----------------------------------------------------------------------------%
\chapter*{\kataPengantar}
%-----------------------------------------------------------------------------%

The basis of this thesis originally stemmed from my passion for open-source
software development. Open-source software has allowed me to tremendously
learn and grow as a computer scientist. Through open-source software, I could
study how pieces of software come together and form state-of-the-art tools that
are widely used by other people. It has opened me to various opportunities in
life that I otherwise could not achieve. Therefore, I believe that it is my
duty to contribute back to the open-source community by sharing the work I have
done on one of my favorite open-source software, the Django web framework.

Praise be to Allah, for the blessing and mercy given to me throughout my study
so that I can finally finish this thesis.

I would like to thank my thesis advisors, \pembimbingSatu\ and \pembimbingDua,
for their advice, support, and guidance throughout the writing of this thesis.

I would also like to thank my academic advisor, Siti Aminah S.Kom., M.Kom.,
and all the lecturers of the Faculty of \fakultas, Universitas Indonesia for
the support, knowledge, and wisdom they have given me during my study. The same
also goes for the teachers at SMAN 48 Jakarta, as I would not be here if it
wasn't for them.

This thesis would not be possible without the continuous support and prayers
from my family and friends. Therefore, I am very grateful for my beloved
parents, Lilis Herlinawati and Muhtadin, and my sisters, Raisha Abdillah and
Khaira Abdillah. Thanks are also due to my friends from high school and
Fasilkom UI, who have supported me since before I begin my undergraduate study.

I understand that this thesis is not perfect. Therefore, should there be any
criticisms, suggestions, and/or questions regarding this thesis, please send
them directly to me via
\href{mailto:sage.muhammad@ui.ac.id}{sage.muhammad@ui.ac.id}. I do hope that
this thesis will be useful for anyone who reads it, especially the students of
the Faculty of \fakultas, Universitas Indonesia and the Django community.

\vspace*{0.1cm}
\begin{flushright}
Depok, \tanggalSiapSidang\\[0.1cm]
\vspace*{1cm}
\penulis

\end{flushright}
