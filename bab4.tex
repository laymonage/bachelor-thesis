%-----------------------------------------------------------------------------%
\chapter{\babEmpat}
%-----------------------------------------------------------------------------%

This chapter explains the implementation of \code{JSONField} and
\code{JSONField}-specific lookups and transforms that can be used on all
database backends supported by Django.

%-----------------------------------------------------------------------------%
\section{\code{JSONField}}
%-----------------------------------------------------------------------------%

In the Django codebase, model fields are defined in the
\code{django.db.models.fields} module as subclasses of \code{Field}, but they
can be imported from the \code{django.db.models} module for simplicity. Model
fields that are relatively simple are implemented in the \verb|__init__.py|
file of the module. For model fields that require more complex implementation
(e.g. file fields and relational fields), they are defined in their own files
in the \code{fields} directory. Due to the complexity of \code{JSONField} and
its extended querying capabilities, the implementation is defined in the
\code{json.py} file in the \code{fields} directory.

\listing
{Python}
{The \code{get\_prep\_value()} method of \code{JSONField} model field.}
{code:jsonfield-ser}
{codes/4-jsonfield-ser.py}

In the previous chapter, \code{JSONField} is designed to serialize Python
objects into JSON-encoded strings by utilizing the built-in \code{json} library
in Python. The serialization process is needed in order to store Python objects
as JSON in the database. To convert Python objects to query values, the
\verb|get_prep_value()| method should be overridden. The serialization is shown
by line 7 of \autoref{code:jsonfield-ser}, where the Python object is passed
to the \verb|json.dumps()| function, which also accepts an optional encoder
class as the \verb|cls| argument. Lines 5 and 6 of the listing show that the
\verb|None| value should not be serialized, because it is reserved for SQL
\verb|NULL| as previously explained.

\listing
[0 pt]
{Python}
{The \code{from\_db\_value()} method of \code{JSONField} model field.}
{code:jsonfield-des}
{codes/4-jsonfield-des.py}

\verb|JSONField| is also designed to deserialize JSON-encoded strings to Python
objects in order to load the JSON data from the database. To convert database
values to Python objects, the \verb|from_db_value()| method can be overridden.
The deserialization process is shown by line 8 of \autoref{code:jsonfield-des},
where the database value is passed to the \verb|json.loads()| function, which
also accepts an optional decoder class as the \verb|cls| argument. Similar to
\verb|get_prep_value()|, the \verb|None| value is reserved for SQL \verb|NULL|,
as shown by lines 5 and 6 of \autoref{code:jsonfield-des}.

When extracting JSON scalar values, the database systems return them as
deserialized values. Passing deserialized values to \verb|json.loads()| may
cause \verb|json.JSONDecodeError| to be raised. Therefore, as shown by lines 7
to 10 of \autoref{code:jsonfield-des}, the \verb|json.loads()| call is put
inside a \verb|try ... except| block and the value is directly returned if
\verb|json.JSONDecodeError| is raised.

\listing
{Python}
{Registration of a stub function for JSONB converter of Psycopg 2.}
{code:defaultjsonb}
{codes/4-defaultjsonb.py}

To allow custom decoder for \verb|JSONField| on PostgreSQL, Psycopg 2's
automatic deserialization of JSON data should be disabled. Disabling the
feature can be done by casting the data to \verb|TEXT| or registering a stub
function for Psycopg 2's JSONB converter \cite{psycopg2:json-adaptation}.
Initially, the casting option was chosen during this research. After the
release of Django 3.1, a bug was found as a result of this option, so the
Django developers have opted to register a stub function \cite{ticket_31956}.
The registration is done in the \verb|get_new_connection()| method of the
database wrapper for PostgreSQL, which gets called every time a new connection
is created. As shown by lines 8 to 10 of \autoref{code:defaultjsonb}, the
\verb|loads| argument is a stub lambda function that returns the parameter
as-is.

\listing
{Python}
{The \code{\_\_init\_\_()} method (constructor) of \code{JSONField}
model field.}
{code:jsonfield-init}
{codes/4-jsonfield-init.py}

The custom encoder and decoder are stored as instance attributes of the field,
as shown by lines 16 and 17 of \autoref{code:jsonfield-init}. It is important
to note that the encoder and decoder are subclasses of \verb|json.JSONEncoder|
and \verb|json.JSONDecoder|, rather than instances of those classes. In Python,
a class is a callable that returns an instance of the class, so the encoder and
decoder have to be callables. To enforce this requirement, the \verb|JSONField|
constructor checks whether the encoder and decoder are callables and raises
descriptive error messages if they are not, as shown by lines 8 to 15 of the
listing.

\listing
{Python}
{The \code{deconstruct()} method of \code{JSONField} model field.}
{code:jsonfield-dec}
{codes/4-jsonfield-dec.py}

In order to preserve the encoder and decoder classes in database migrations,
the \verb|deconstruct()| method needs to be overridden. The method returns a
4-tuple that consists of: the name of the field on the model, the import path
of the field, a list of positional arguments, and a dictionary of keyword
arguments \cite{django:model_fields}. These values will be used to reconstruct
the field from a database migration. The encoder and decoder are defined as
keyword arguments of the constructor. Therefore, the \verb|deconstruct()|
method is overridden to add the encoder and decoder classes (if they are set,
i.e. not \verb|None|) to the keyword arguments dictionary, as shown by lines 6
to 9 of \autoref{code:jsonfield-dec}.

\listing
{Python}
{The \code{\_\_init\_\_()} method (constructor) of \code{JSONField} form field.}
{code:formfield-init}
{codes/4-formfield-init.py}

The \verb|JSONField| form field has also been updated to support custom encoder
and decoder. As with the model field, the encoder and decoder classes are
stored as instance attributes of the form field, as shown by lines 8 and 9 of
\autoref{code:formfield-init}. The encoder and decoder classes are used in
\verb|json.dumps()| and \verb|json.loads()| calls throughout the form field,
respectively.

\listing
{Python}
{The \code{prepare\_value()} and \code{has\_changed()} methods of
\code{JSONField} form field.}
{code:formfield-dumps}
{codes/4-formfield-dumps.py}

To utilize the custom encoder, some of the \verb|JSONField| form field methods
had to be updated. Line 7 of \autoref{code:formfield-dumps} shows the encoder
class used in the \verb|json.dumps()| call inside the \verb|prepare_value()|
method, which is used to prepare the value before it is shown to the user (e.g.
in an HTML form). Lines 15 and 16 show the encoder class used in the
\verb|json.dumps()| calls inside the \verb|has_changed()| method, which is used
to check whether the value inside the form field has changed.

\listing
[0 pt]
{Python}
{The \code{to\_python()} and \code{bound\_data()} methods of \code{JSONField}
form field.}
{code:formfield-loads}
{codes/4-formfield-loads.py}

Some of the \verb|JSONField| form field methods also had to be updated to
support the use of a custom decoder. Line 18 of \autoref{code:formfield-loads}
shows the decoder class used in the \verb|json.loads()| call inside the
\verb|to_python()| method, which is used in the form field validation process.
If the value cannot be deserialized with the decoder, a \verb|ValidationError|
is raised, as shown by lines 20 to 24. Line 34 shows the decoder class used in
the \verb|json.loads()| call inside the \verb|bound_data()| method, which is
used to load the input data to the form field.

The new form field implementation remains largely unchanged from the existing
implementation. As explained above, the implementation changes were only made
so that the form field has custom encoder and decoder support. Other than that,
the form field has now been moved from the \verb|django.contrib.postgres.forms|
module to the \verb|django.forms| module. The reason behind the small amount of
changes is the fact that the form field only handles user input and does not
interact with the database directly, which means it's already possible to use
it with any database backend.

\listing
{Python}
{The \code{formfield()} method of \code{JSONField} model field.}
{code:jsonfield-form}
{codes/4-jsonfield-form.py}

When model fields are included in a \verb|ModelForm|, Django automatically
generate form fields for them \cite{django:modelform}. The form fields are
generated by calling the \verb|formfield()| method of the model fields. In
order to use the encoder and decoder from the \verb|JSONField| model field in
the \verb|JSONField| form field, they need to be passed as keyword arguments,
as shown by lines 7 and 8 of \autoref{code:jsonfield-form}.
