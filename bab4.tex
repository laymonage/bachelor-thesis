%-----------------------------------------------------------------------------%
\chapter{\babEmpat}
%-----------------------------------------------------------------------------%

This chapter explains the implementation of \code{JSONField} and
\code{JSONField}-specific lookups and transforms that can be used on all
database backends supported by Django.

%-----------------------------------------------------------------------------%
\section{Database Backend Adjustments}
%-----------------------------------------------------------------------------%

Before implementing the \verb|JSONField| class, it is important to remember
that there are differences between the database systems in regards to handling
JSON data. For example, each database system uses a different data type to
store JSON data. In addition, the database driver (e.g. Psycopg 2 for
PostgreSQL) may automatically deserialize JSON data into Python objects.

Django keeps track of the database systems' differences in the
\verb|django.db.backends| module. Each database system has its own database
backend submodule.\footnote{Except for MariaDB and MySQL, as they can share the
same driver and backend.} The database backends' classes are extended from the
classes in the \verb|django.db.backends.base| submodule. For example, the
\verb|DatabaseFeatures| class stores boolean flags and other feature-related
information as the class' attributes. Default values for these attributes are
defined in the \verb|BaseDatabaseFeatures| class.

\listing
{Python}
{The three additional flags in the \code{BaseDatabaseFeatures} class.}
{code:basedbfeatures}
{codes/4-basedbfeatures.py}

To implement the \verb|JSONField| class, three flags are added to the
\verb|BaseDatabaseFeatures| class. The first flag is
\verb|supports_json_field|, which shows whether the database system supports
\verb|JSONField|. The second flag is \verb|supports_primitives_in_json_field|,
which is used to determine whether the database system supports storing JSON
scalar values (strings, numbers, booleans, and \verb|null|) directly in a
\verb|JSONField|.\footnote{This flag is not used in the \code{JSONField}
implementation itself, but it is useful to store this information to be used in
tests.} The third flag is \verb|has_native_json_field|, which shows whether the
database system has a native JSON data type and automatically deserializes JSON
data as Python objects. These flags are overridden in each database backend's
\verb|DatabaseFeatures| class as necessary.

\listing
[0 pt]
{Python}
{The \code{DatabaseWrapper} and \code{DatabaseFeatures} classes
of the PostgreSQL backend.}
{code:backends-postgresql}
{codes/4-backends-postgresql.py}

\autoref{code:backends-postgresql} shows the \verb|DatabaseWrapper| and
\verb|DatabaseFeatures| classes of the PostgreSQL backend after the
modification needed for \verb|JSONField|. The \verb|data_types| property is
updated to map \verb|JSONField| to the \verb|jsonb| data type, as shown by line
7 of the listing. The \verb|jsonb| data type automatically checks whether the
data is valid JSON, so there is no need for a \verb|CHECK| constraint.
PostgreSQL has a native JSON data type and the driver (Psycopg 2) automatically
deserializes JSON data into Python objects, so the \verb|has_native_json_field|
is set to \verb|True|, as shown by line 4 of the listing. The other flags
(\verb|supports_json_field|\footnote{Django supports PostgreSQL 9.5 and higher,
while the \code{jsonb} is available from version 9.4, so \code{JSONField} is
always supported.} and \verb|supports_primitives_in_json_field|) are not
overridden, as their values from \verb|BaseDatabaseFeatures| are already
correct.

\listing
{Python}
{The \code{DatabaseWrapper} class of the MySQL backend.}
{code:backends-mysql-1}
{codes/4-backends-mysql-1.py}

The MySQL database backend, which is used for MariaDB and MySQL, is modified to
add support for \verb|JSONField|. For this backend, the \verb|DatabaseWrapper|
class is modified by adding a mapping from \verb|JSONField| to the \verb|json|
data type in the \verb|data_types| property, as shown by line 6 of
\autoref{code:backends-mysql-1}. The \verb|data_type_check_constraints|
property is also modified to add the \verb|JSON_VALID()| function as a
\verb|CHECK| constraint for MariaDB prior to version 10.4.3, as shown by lines
15 and 16 of the listing. Later versions of MariaDB have the \verb|CHECK|
constraint automatically enabled when using the \verb|JSON| data type alias,
while MySQL automatically validates the JSON syntax when using the \verb|JSON|
data type. Therefore, there is no need to modify
\verb|data_type_check_constraints| for these systems.

\listing
{Python}
{The \code{DatabaseFeatures} class of the MySQL backend.}
{code:backends-mysql-2}
{codes/4-backends-mysql-2.py}

Meanwhile, the \verb|DatabaseFeatures| class is modified by turning
\verb|supports_json_field| into a cached property method. The method returns
\verb|True| for MariaDB version 10.2.7 or greater and MySQL version 5.7.8 or
greater (and \verb|False| otherwise), as shown by lines 4 to 8 of
\autoref{code:backends-mysql-2}. The other flags
(\verb|supports_primitives_in_json_field| and \verb|has_native_json_field|)
use the values from \verb|BaseDatabaseFeatures| as they are already correct.

\listing
{Python}
{The \code{DatabaseWrapper} class of the Oracle Database backend.}
{code:backends-oracle-1}
{codes/4-backends-oracle-1.py}

For the Oracle Database backend, the \verb|DatabaseWrapper| class is modified
to add the data type mapping and \verb|CHECK| constraint. The \verb|data_types|
property is updated to map \verb|JSONField| to \verb|NCLOB|, as shown by line 7
of \autoref{code:backends-oracle-1}. The \code{NCLOB} data type is chosen to
allow data larger than 32,767 bytes and to have Unicode support
\cite{oracle:overview-json}. The \verb|data_type_check_constraints| property is
also updated to apply the \verb|IS JSON| constraint for \verb|JSONField|, as
shown by line 12 of the listing.

\listing
{Python}
{The \code{DatabaseOperations} class of the Oracle Database backend.}
{code:backends-oracle-2}
{codes/4-backends-oracle-2.py}

The cx\_Oracle database driver returns \verb|CLOB| and \verb|BLOB| values from
the database as \verb|LOB| objects in Python \cite{cxoracle:lob}. Meanwhile,
the \verb|LOB| objects need to be converted to Python strings in order to
perform deserialization. To do this, the \verb|get_db_converters()| method in
the \verb|DatabaseOperations| class is overridden so that \verb|JSONField| uses
the \verb|convert_textfield_value| converter, as shown by lines 7 and 8 of
\autoref{code:backends-oracle-2}.\footnote{\code{TextField} also uses the
\code{NCLOB} data type on Oracle Database, so its converter can be reused for
\code{JSONField}.}

\listing
{Python}
{The \code{DatabaseFeatures} class of the Oracle Database backend.}
{code:backends-oracle-3}
{codes/4-backends-oracle-3.py}

As for the \verb|DatabaseFeatures| class, the \verb|supports_primitives_in_json_field|
flag is overridden to \verb|False|, as shown by line 4 of
\autoref{code:backends-oracle-2}. This flag reflects the behavior of Oracle
Database that only allows storing JSON objects and arrays in a column with the
\verb|IS JSON| constraint enabled. The other flags
(\verb|supports_json_field|\footnote{Oracle Database introduced JSON support in
version 12.1.0.2 and Django supports Oracle Database version 12.2 and higher,
so \code{JSONField} is always supported.} and \verb|has_native_json_field|) are
derived from \verb|BaseDatabaseFeatures|.

\listing
{Python}
{The \code{DatabaseWrapper} class of the SQLite backend.}
{code:backends-sqlite3-1}
{codes/4-backends-sqlite3-1.py}

\listing
[0 pt]
{Python}
{The \code{DatabaseFeatures} class of the SQLite backend.}
{code:backends-sqlite3-2}
{codes/4-backends-sqlite3-2.py}

The \verb|DatabaseWrapper| class of the SQLite backend is modified to add the
mapping from \verb|JSONField| to \verb|text| in the \verb|data_types| property,
as shown by line 7 of \autoref{code:backends-sqlite3-1}. It is also modified to
apply the \verb|JSON_VALID()| check constraint to the database column. As the
function returns \verb|0| (false) for SQL \verb|NULL| values, an
\verb|OR "%(column)s" IS NULL| clause is added to support storing SQL
\verb|NULL| values, as shown by line 12 of the listing.

Meanwhile, \verb|DatabaseFeatures| class is modified to check for JSON support
on SQLite. The \verb|supports_json_field| flag is turned into a cached property
method and the value is determined by trying to execute a \verb|SELECT| query
with the \verb|JSON()| function included in the JSON1 extension, as shown by
lines 4 to 8 of \autoref{code:backends-sqlite3-2}. If the query does not throw
an exception, it means the JSON1 extension is enabled and the flag is set to
\verb|True| (and \verb|False| otherwise), as shown by lines 9 to 11 of the
listing. The support check is done this way because SQLite does not provide a
direct way to check whether the JSON1 extension is loaded. The other flags
(\verb|supports_primitives_in_json_field| and \verb|has_native_json_field|)
are derived from \verb|BaseDatabaseFeatures|.

All of the database backends have now been adjusted with the necessary changes.
The changes were mostly made for mapping \verb|JSONField| to the correct data
type on the database, as well as the \verb|CHECK| constraints to ensure that
the data is valid JSON. With these changes in place, the \verb|JSONField| class
can now be implemented.

%-----------------------------------------------------------------------------%
\section{\code{JSONField}}
%-----------------------------------------------------------------------------%

In the Django codebase, model fields are defined in the
\verb|django.db.models.fields| module as subclasses of \verb|Field|, but they
can be imported from the \verb|django.db.models| module for simplicity. Model
fields that are relatively simple are implemented in the \verb|__init__.py|
file of the module. For model fields that require more complex implementation
(e.g. file fields and relational fields), they are defined in their own files
in the \verb|fields| directory. Due to the complexity of \verb|JSONField| and
its extended querying capabilities, the implementation is defined in the
\verb|json.py| file in the \verb|fields| directory.

\listing
{Python}
{The \code{get\_prep\_value()} method of \code{JSONField} model field.}
{code:jsonfield-ser}
{codes/4-jsonfield-ser.py}

In the previous chapter, \verb|JSONField| is designed to serialize Python
objects into JSON-encoded strings by utilizing the built-in \verb|json| library
in Python. The serialization process is needed in order to store Python objects
as JSON in the database. To convert Python objects to query values, the
\verb|get_prep_value()| method should be overridden. The serialization is shown
by line 7 of \autoref{code:jsonfield-ser}, where the Python object is passed
to the \verb|json.dumps()| function, which also accepts an optional encoder
class as the \verb|cls| argument. Lines 5 and 6 of the listing show that the
\verb|None| value should not be serialized, because it is reserved for SQL
\verb|NULL| as previously explained.

\listing
{Python}
{The \code{from\_db\_value()} method of \code{JSONField} model field.}
{code:jsonfield-des}
{codes/4-jsonfield-des.py}

\verb|JSONField| is also designed to deserialize JSON-encoded strings to Python
objects in order to load the JSON data from the database. To convert database
values to Python objects, the \verb|from_db_value()| method can be overridden.
On some database systems that have a native JSON data type, the database driver
may already deserialize the value before Django receives it.\footnote{Only
PostgreSQL's Psycopg 2 driver does this at the time of writing, but this may
change in the future.} Line 7 and 8 of \autoref{code:jsonfield-des} show that
the value is returned as-is, unless a custom decoder is used. On other database
systems, the deserialization process is shown by line 10 of the listing, where
the database value is passed to the \verb|json.loads()| function, which also
accepts an optional decoder class as the \verb|cls| argument. Similar to
\verb|get_prep_value()|, the \verb|None| value is reserved for SQL \verb|NULL|,
as shown by lines 5 and 6 of the listing.

When extracting JSON string values, the database systems return them as
deserialized values (i.e. without JSON double quotes). Passing deserialized
strings to \verb|json.loads()| may cause \verb|json.JSONDecodeError| to be
raised. Therefore, as shown by lines 7 to 10 of \autoref{code:jsonfield-des},
the \verb|json.loads()| call is put inside a \verb|try ... except| block and
the value is directly returned if \verb|json.JSONDecodeError| is raised.

To allow a custom decoder for \verb|JSONField| on PostgreSQL, Psycopg 2's
automatic deserialization of JSON data should be disabled. Disabling the
feature can be done by casting the data to \verb|TEXT| or registering a stub
function for Psycopg 2's \verb|jsonb| converter \cite{psycopg2:json-adaptation}.
The casting option was chosen during this research, as registering a stub
function may break compatibility with the previous implementation that
relied on the automatic deserialization. As noted on the Psycopg 2
documentation, the casting operation is efficient and does not involve a copy.

\listing
{Python}
{The \code{select\_format()} method of \code{JSONField}.}
{code:selectformat}
{codes/4-selectformat.py}

\listing
{Python}
{The \code{json\_cast\_text\_sql()} method of the \code{DatabaseOperations}
class in Django's PostgreSQL database backend.}
{code:pgops}
{codes/4-pgops.py}

To cast JSON data to \verb|TEXT| on the database level, the
\verb|select_format()| method is overridden in the \verb|JSONField| class. As
the name suggests, this method determines the format of the \verb|SELECT|
clause of the SQL query. The casting is done only if a custom decoder is used
with a database backend that has a native JSON data type (and automatically
deserializes it), as shown by lines 5 to 9 of \autoref{code:selectformat}.
In the case of PostgreSQL, casting data to \verb|TEXT| can be done using the
\verb|::text| syntax, as shown by \autoref{code:pgops}.

\listing
{Python}
{The \code{\_\_init\_\_()} method (constructor) of \code{JSONField}
model field.}
{code:jsonfield-init}
{codes/4-jsonfield-init.py}

The custom encoder and decoder are stored as instance attributes of the field,
as shown by lines 16 and 17 of \autoref{code:jsonfield-init}. It is important
to note that the encoder and decoder are subclasses of \verb|json.JSONEncoder|
and \verb|json.JSONDecoder|, rather than instances of those classes. In Python,
a class is a callable that returns an instance of the class, so the encoder and
decoder have to be callables. To enforce this requirement, the \verb|JSONField|
constructor checks whether the encoder and decoder are callables and raises
descriptive error messages if they are not, as shown by lines 8 to 15 of the
listing.

\listing
{Python}
{The \code{deconstruct()} method of \code{JSONField} model field.}
{code:jsonfield-dec}
{codes/4-jsonfield-dec.py}

In order to preserve the encoder and decoder classes in database migrations,
the \verb|deconstruct()| method needs to be overridden. The method returns a
4-tuple that consists of: the name of the field on the model, the import path
of the field, a list of positional arguments, and a dictionary of keyword
arguments \cite{django:model_fields}. These values will be used to reconstruct
the field from a database migration. The encoder and decoder are defined as
keyword arguments of the constructor. Therefore, the \verb|deconstruct()|
method is overridden to add the encoder and decoder classes (if they are set,
i.e. not \verb|None|) to the keyword arguments dictionary, as shown by lines 6
to 9 of \autoref{code:jsonfield-dec}.

\listing
{Python}
{The \code{\_\_init\_\_()} method (constructor) of \code{JSONField} form field.}
{code:formfield-init}
{codes/4-formfield-init.py}

The \verb|JSONField| form field has also been updated to support custom encoder
and decoder. As with the model field, the encoder and decoder classes are
stored as instance attributes of the form field, as shown by lines 8 and 9 of
\autoref{code:formfield-init}. The encoder and decoder classes are used in
\verb|json.dumps()| and \verb|json.loads()| calls throughout the form field,
respectively.

\listing
{Python}
{The \code{prepare\_value()} and \code{has\_changed()} methods of
\code{JSONField} form field.}
{code:formfield-dumps}
{codes/4-formfield-dumps.py}

To utilize a custom encoder, some of the \verb|JSONField| form field methods
had to be updated. Line 7 of \autoref{code:formfield-dumps} shows the encoder
class used in the \verb|json.dumps()| call inside the \verb|prepare_value()|
method, which is used to prepare the value before it is shown to the user (e.g.
in an HTML form). Lines 15 and 16 show the encoder class used in the
\verb|json.dumps()| calls inside the \verb|has_changed()| method, which is used
to check whether the value inside the form field has changed.

\listing
{Python}
{The \code{to\_python()} and \code{bound\_data()} methods of \code{JSONField}
form field.}
{code:formfield-loads}
{codes/4-formfield-loads.py}

Some of the \verb|JSONField| form field methods also had to be updated to
support the use of a custom decoder. Line 18 of \autoref{code:formfield-loads}
shows the decoder class used in the \verb|json.loads()| call inside the
\verb|to_python()| method, which is used in the form field validation process.
If the value cannot be deserialized with the decoder, a \verb|ValidationError|
is raised, as shown by lines 20 to 24. Line 34 shows the decoder class used in
the \verb|json.loads()| call inside the \verb|bound_data()| method, which is
used to load the input data to the form field.

The form field only handles user input and does not interact with the database
directly. Thus, it is possible to use it with any database backend. As a
result, the form field has now been moved from the
\verb|django.contrib.postgres.forms| module to the \verb|django.forms| module.

\listing
{Python}
{The \code{formfield()} method of \code{JSONField} model field.}
{code:jsonfield-form}
{codes/4-jsonfield-form.py}

When model fields are included in a \verb|ModelForm|, Django automatically
generate form fields for them \cite{django:modelform}. The form fields are
generated by calling the \verb|formfield()| method of the model fields. In
order to use the encoder and decoder from the \verb|JSONField| model field in
the \verb|JSONField| form field, they need to be passed as keyword arguments,
as shown by lines 7 and 8 of \autoref{code:jsonfield-form}.
