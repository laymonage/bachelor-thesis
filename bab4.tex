%-----------------------------------------------------------------------------%
\chapter{\babEmpat}
%-----------------------------------------------------------------------------%

This chapter explains the implementation of \code{JSONField} and
\code{JSONField}-specific lookups and transforms that can be used on all
database backends supported by Django.

%-----------------------------------------------------------------------------%
\section{\code{JSONField}}
%-----------------------------------------------------------------------------%

In the Django codebase, model fields are defined in the
\code{django.db.models.fields} module. Model fields that are relatively simple
are implemented in the \verb|__init__.py| file of the module. For model fields
that require more complex implementation, they are defined in their own files in
the \code{fields} directory. Due to the complexity of \code{JSONField} and its
extended querying capabilities, the implementation is defined in the
\code{json.py} file in the \code{fields} directory.

In the previous chapter, \code{JSONField} is designed to serialize Python
objects and deserialize JSON-encoded strings by utilizing the built-in
\code{json} library in Python. The serialization process is needed in order to
store Python objects as JSON in the database. Meanwhile, the deserialization
process is needed in order to load the JSON data from the database as Python
objects.



%-----------------------------------------------------------------------------%
\section{laporan\_setting.tex}
%-----------------------------------------------------------------------------%
Berkas ini berguna untuk mempermudah pembuatan beberapa template standar.
Anda diminta untuk menuliskan judul laporan, nama, npm, dan hal-hal lain yang dibutuhkan untuk pembuatan template.


%-----------------------------------------------------------------------------%
\section{istilah.tex}
%-----------------------------------------------------------------------------%
Berkas istilah digunakan untuk mencatat istilah-istilah yang digunakan.
Fungsinya hanya untuk memudahkan penulisan.
Pada beberapa kasus, ada kata-kata yang harus selalu muncul dengan tercetak miring atau tercetak tebal.
Dengan menjadikan kata-kata tersebut sebagai sebuah perintah \latex~tentu akan mempercepat dan mempermudah pengerjaan laporan.


%-----------------------------------------------------------------------------%
\section{hype.indonesia.tex}
%-----------------------------------------------------------------------------%
Berkas ini berisi cara pemenggalan beberapa kata dalam bahasa Indonesia.
\latex~memiliki algoritma untuk memenggal kata-kata sendiri, namun untuk beberapa kasus algoritma ini memenggal dengan cara yang salah.
Untuk memperbaiki pemenggalan yang salah inilah cara pemenggalan yang benar ditulis dalam berkas \f{hype.indonesia.tex}.


%-----------------------------------------------------------------------------%
\section{pustaka.tex}
%-----------------------------------------------------------------------------%
Berkas pustaka.tex berisi seluruh daftar referensi yang digunakan dalam
laporan.
Anda bisa membuat model daftar referensi lain dengan menggunakan bibtex.
Untuk mempelajari bibtex lebih lanjut, silahkan buka \url{http://www.bibtex.org/Format}.
Untuk merujuk pada salah satu referensi yang ada, gunakan perintah \bslash cite, e.g. \bslash cite\{book:sample\} yang akan akan memunculkan \cite{book:sample}.


%-----------------------------------------------------------------------------%
\section{bab[1 - 6].tex}
%-----------------------------------------------------------------------------%
Berkas ini berisi isi laporan yang Anda tulis.
Setiap nama berkas e.g. bab1.tex merepresentasikan bab dimana tulisan tersebut akan muncul.
Sebagai contoh, kode dimana tulisan ini dibaut berada dalam berkas dengan nama \code{bab4.tex}.
Ada enam buah berkas yang telah disiapkan untuk mengakomodir enam bab dari laporan Anda, diluar bab kesimpulan dan saran.
Jika Anda tidak membutuhkan sebanyak itu, silahkan hapus kode dalam berkas \code{thesis.tex} yang memasukan berkas \latex~yang tidak dibutuhkan; contohnya perintah \code{\bslash{}include\{bab6.tex\}} merupakan kode untuk memasukan berkas \code{bab6.tex} kedalam laporan.

